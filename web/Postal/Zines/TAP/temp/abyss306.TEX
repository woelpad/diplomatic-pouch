\nopagenumbers 
\font\twelverm=cmbx12 at 12pt
\centerline{\bf\twelverm THE ABYSSINIAN PRINCE \#306}
\line{\hfil October 16, 2006}    
\smallskip                                                    
\settabs 3 \columns 
\beginsection

\medskip

Produced by Jim Burgess, 664 Smith Street, Providence, RI 02908-4327  USA,
(401)351-0287 
 
\noindent Accessible through Internet at burgess of world.std.com (all E-Mail
addresses are reported in this format, replace the `` of '' with ``@''; if you 
bounce try sending to me from another account. 
Some of you have been getting bounced messages from
my ISP's spam protection, if that happens to you, USE my backup E-Mail
at jfburgess of gmail.com!!!  Don't complain that my E-Mail keeps bouncing
without forwarding the bounced message to that address.
Then I can forward it to the ISP help to get it dealt with.
 
\noindent Web Page Address: /Postal/Zines/TAP/index.html

\bigskip

The szine is late again, I am having job issues that could necessitate other such
delays in the near future.
I pushed deadlines to account for this, and will keep doing that as necessary.
JOIN the 23 TUNES GAME!!!  SEND ME THREE TUNES AND COMMENTARY!!!

Here are the rules for 23 TUNES.  You send me three tunes for the first turn,
and then two tunes in each of the last ten turns for a total of 23.
I will also be submitting my tunes.
After we're done, I'd like to exchange CD's/Tapes for as many of the tunes players
as possible, but this is not required.
I'll be sending the winner both my LAST 23 tune list and my NEW 23 tune list.
The winner is determined by having you guess each issue who submitted what list
(I will tell you who the submitters are).  For each list you get right, you get
a point, you also can win bonus points from me for really cool tune selections.
That's it, not complicated.
I think I have three lists so far, I'd like to play with between 5 and 10 of you,
so let's get started!!!!

I expect ALL players to be signed up on the E-Mail notification list for the szine,
see below.
Some of you have been complaining about this, but it is up to you to get on this list,
it's easy, come ask me if you have trouble.
We have another exciting issue of {\it By the Way} in this issue if everything
is going as planned.
We're making progress on starting new games, help fill things up, see below for more.
The new regular Diplomacy game has started, Spring 1901 results in this issue!!!

\bigskip

%Dave Partridge's unlisted phone number 603-672-2879 -- probably corrected!!!
%Tinamou players: Schwarzschild:  Graham Wilson (NO 
%but E-mails: grahamaw of home.com (or gwilson of dvnet.com)),
%Berry Renken, Phil Reynolds, Heath Gardner
%Copernicus:  Warren Goesle, Heath Gardner, Art Schleinkofer, Bruce Reiff,
%Eric Brosius, Bob Dowrey
%Tour De Farce:  Jim Burgess, Eric Brosius, Warren Goesle, Sara Reichert,
%Phil Reynolds, Rick Desper
%Milk and Cookies: Jim Burgess, Warren Goesle, Sara Reichert, Phil Reynolds,
%Rick Desper, Richard Weiss, John Schultz
%Who's Your Buddy:  Sara Reichert, Phil Reynolds, Art Schleinkofer, 
%Brad Wilson, Rick Copeland, Karl Muller

The postal sub price is still
\$1.50 per issue in the US and Canada,
with double that for other foreign subbers(or \$3.00 per
issue sent airmail).
Players in current games and standbys will
continue to get the issues for free, and future game starts 
(except for Nuclear Yuppie Evil Empire Diplomacy, which is free) cost 
\$20.00 (\$15.00 for a life of the
game subscription and \$5 for the NMR Insurance.
Anyone may play in subszines 
for free and just jack up the issue page count. 
See the revised game start announcements below!

Check out the connections in the Diplomatic Pouch with all of
the information you need to play Diplomacy on the Internet at:
/

I also have taken over the Postal port/DipPouch/Postal

\noindent and {\it TAP} on the web is there at:
/Postal/Zines/TAP/index.html

where the szine resides in html format. 
Presently, issues from \#190 to the current issue
are there, and I will be updating the back issues gradually someday. 
Also, check out Stephen Agar's more extensive efforts at:
http://www.diplomacy-archive.com and http://www.diplomacy.co.uk
 
Peter Sullivan's subszine is out of stasis, and all the
back issues can be accessed via :

http://www.burdonvale.co.uk/octopus/index.html.

Peter now seems to be sliding back into stasis!
Rip Gooch and Dave Partridge had been picking up the choo-choo
game slack,
but Rip also has been missing in action completely for over a year now, 
I don't have an update but he
and Dave Partridge communicate. 
Contact Rip at xyropedes of canada.com or Dave
at rebhuhn of rocketmail.com for more info on getting into choo-choo games.
And Andy York has been my most frequent subszine guest lately!!!
He's been in every issue, thanks, Andy!!!

The {\it TAP} mailing list has moved!
It now is even BETTER protected than it was.
I even have a bit of trouble posting to it.
To post to this list, send your email to: tap of diplom.org.
But this is completely moderated, it won't go out to the list unless
I approve it.
In general, I intend to keep traffic down to just the szine, as we've
been doing and I'll put your LOCs in here.
I {\bf EXPECT} all players to be on this list, especially those of you who
are from foreign countries!!! 
You need this to find out when the szine is up on the web to check in on results.

General information about the mailing list is at:
/mailman/listinfo/tap

You can sign up from there, or send E-Mails to:
Tap-request of diplom.org;
with the word `help' in the subject or body (don't include the
quotes), and you will get back a message with instructions.
You must know your password to change your options (including changing
the password, itself) or to unsubscribe.  
Normally, Mailman will remind you of your diplom.org mailing list
passwords once every month, although you can disable this if you
prefer.  This reminder will also include instructions on how to
unsubscribe or change your account options.  There is also a button on
your options page that will email your current password to you.
A big, big thank you for Millis Miller for setting this all up!!

\bigskip

\centerline{\bf THE SEARCH FOR MARGARET GEMIGNANI}

\smallskip

OK, now, why Margaret?  Margaret, of course, is one of the most famous personalities in 
Dipdom, Science Fiction, Star Trek, Comic, etc. fandom.  
She was a nurse in the DC area, I believe, and then retired to Florida some years ago.
Many people had an address for her in Fort Lauderdale (including me), but that address
went missing around 1994.
If you do Google searching, as I have done off and on, there is a Margaret Gemignani
that contributes to the Baltimore and Ohio Historical Railroad Association, along with 
a husband, Gino (anyone know if that matches.... the B\&O matches where she used to live...
but don't tell me, use it as info to go looking yourself!!!)
Margaret has a couple of mentions in the Diplomacy AtoZ and I first encountered her
in the old Bernie Oaklyn {\it Le Front} szine.
I think it would be fun to find her.

\bigskip

Feel free to spend the time looking for some of the backlog.
Let's get Jeff, Derek, Sylvain, Steve, Ed, Tom, Bill, Gregory, and 
ESPECIALLY Kevin found too!!!
This is a regular continuing feature of the szine
and I will be introducing a new ``search for'' every five issues.
Moreover, you can win a \$25 prize for finding some previous
target who went unfound in the original \$50 period.
That means that if Jeff Key or Steve Heinowski or Ed Henry or Tom Hurst or Bill Quinn or 
Gregory Stewart or Derek Nelson or Sylvain LaRose or John Smythe
is ``found'' from now on it is worth \$25.

\bigskip

Winners will receive credit for Dip hobby activities
that I will pay out as requested by the winner.
Subscribe to szines here or abroad, run your
own contests, publish a szine, finance a web page, 
GO TO A DIPLOMACY CONVENTION or whatever.
Spend it all right away or use me as a bank to cover hobby activities
for years.
What must you do to win?
Get me a letter to the editor for {\it TAP} from the person we're
searching for.

This is very important, just finding them doesn't do it.
They have to write me a letter.     
The final judge as to the winner of any contest will be the target 
himself and I reserve the right to investigate the winning entry.
When you find someone I'm looking for, you should ask him to send me a 
letter for print that includes a verification of who ``found'' him.
%Michael Swift, Mike Stewart, and 
%Jason Yarbrough , Tim Stabosz, 
%Eric Ozog has $25
%Leslie Obata is owed $25
%Derek Nelson, winner of the first postal game in Graustark.  I last saw him
%at the '76 DipCon in Baltimore.
%Charles Wells, former publisher
%John McCallum, former publisher (Brobdignag, SerenDip, etc.)
%Dan/Charles Brannan, former publisher (Wild 'n Wooly)
%Margaret Gemignani, former player and nurse living in Florida

\bigskip

{\bf Cal White (Tue, 3 Oct 2006 09:17:30 -0400)}

Hi Jim,
Just wanted to let you know that I did a quick check of area phone books 
looking for Derek Nelson.  I called the only one listed, but it was not the 
right one.  I'd like to see you find Derek because he's the guy I rode with 
from Toronto to Baltimore to attend DipCon 76 at Johns Hopkins.  Never knew 
him well (that was the only time we had contact), but he was a nice guy.

\centerline{Cal White, diplomat of idirect.com}

{\it ((I always welcome comments like ``I looked here and found that"
types of things.
I don't ordinarily print them unless you say to (which Cal did) in case
you want to keep your search info private.
Let's get these guys found, I've heard that some other people had been looking.))}

\bigskip

\centerline{\bf INTERNATIONAL SUBSCRIPTION EXCHANGE NEWS}

\smallskip

The British representative is the editor of
{\it Mission From God}, John Harrington.
John may be contacted at 1 Churchbury Close, Enfield, Middlesex
EN1 3UW, UK.

E-Mail: fiendish of operamail.com, John.Harrington of tfeurope.com 

Please include the full name and address of the foreign publisher with
your order, if possible, as well as the szine title.
Make your check in US dollars out to me personally or in GBP to John
if you're doing things from that end.
I will conduct business for Canadians as well, if I can, but prefer
to deal in US dollars with them if possible, or Canadian dollars cash.
To subscribe to American szines, the system works in reverse.

We have closed the European continental branch, as I think most
of you had figured out.

And the ISE in Australia hadn't had much real action in recent
years, and Brendan Whyte has moved on to Jerusalem!!!
Brendan still produces what I find to be the most readable
small szine in the worldwide hobby.
Did you all realize that?
Write to Brendan at his new Jerusalem address and ask him
about subscribing, I'm not sure what the new deal will be.
Department of Geography, Faculty of Social Sciences,
The Hebrew University of Jerusalem, Mount Scopus, Jerusalem
91905, ISRAEL.
His travelogues are wonderful.

\bigskip  

\centerline{\bf WORLDMASTERS04 SECTION}

\smallskip

See http://www.worldmasters.net/wm04/
for details on progress on the WORLDMASTERS04, the Semifinals have now begun.  
Some semifinal notables include: Egg Ferreira, Buffalo Bartalone, Lee Simpson, 
Joe Janbu, Sebastian Beer,
Jerry Fest, Twerg O'Donnchu, Tim Sweeney, Dave Partridge, Glenn Ledder,
Adam Silverman, Toby Harris, and Thomas Franke.
This is one of the deepest semifinal fields for this great tournament 
that I've seen, with representatives from nearly all of the world's
hobby communities.
Actually, in an update, the Semifinal games will begin in Mid-September.
I was wondering why I hadn't seen anything on progress in these games,
and that's why.
But the pairings have just been announced.
More on that next issue!!!
Thomas Franke and David Partridge are in a game together that seems
on first glance to be the weakest in the rest of the players.
So my first thought is that one of those two will have a slot in
the finals.
I am getting raked over the coals for it, but I just ended a game
in a three way draw between me, Thomas, and Ray Setzer that I might
have had a chance in ekeing out a win mostly because Thomas asked
for it to end so he could focus on this semi-final.
Sorry, David, but I had to do it.....

It is helping now that I've joined the Yahoogroup WM04-Chat where this
discussion is happening.
You can join too!
However, this forum has been very, very quiet, I hope it picks up with
semis, maybe I'll have to start some discussion.
But most of the talk that there has been has about Yann Clouet and the French
Hobby's new ``World Palmares Evaluation" of FTF Diplomacy Play.
Of course, Yann Clouet comes out \#1, further backing up winning
the John Koning award, and I am 1442nd (in a tie with Mark Nelson
and Tony Dickinson, among others!).  
This uses lots of tournament results, more than 750 of them, and really
is quite comprehensive.
See those rankings at:
http://www.18centres.com/SPIP3/article.php3?id\_article=199 

\bigskip

\centerline{\bf DIPDOM NEWS SECTION (with letters)}

\smallskip
 
Obscure and not-so-obscure ramblings on the state of the hobby
and its publications, custodians, events, and individuals with
no guarantee of relevance from the fertile keyboard of Jim-Bob,
the E-Mail Dip world, and the rest of the postal hobby.
My comments are in {\it italics} and {\it ((double 
quotation marks))} like this.
{\bf Bold face} is used to set off each individual speaker.
I should also make a note that I do edit for syntax and spelling
on occasion.

%A VERY light discussion is taking place that will address what 
%stance we (the hobby) should take (proactive in some way for sure) 
%toward Hasbro, the new owners of the rights to Diplomacy.
%More on this will be forthcoming, but it looks like Hasbro might
%have its version of Diplomacy on the market for the Christmas
%season and they actually will be employing ``play-testers''.
%No word on precisely what this means yet, could it mean that
%they will offer a series of ``official'' variants?
%Stay tuned....
%If you want to be part of the discussion, send the MESSAGE:
%
%subscribe hasbro
%
%to majordomo of diplom.org, it works just like the tap mailing list
%described below.  Sending messages to hasbro of www.diplom.org
%sends the mail to the whole list.
%The big news this time is the beginning of the breakout of
%Hasbro into official contact with the hobby.

The game Diplomacy is a copyrighted product owned by Hasbro and all
reproductions or other use of that material in this szine is intended to
be personal use and not infringe on those rights in any way.
All reproductions are done at a heavy financial loss to the editor
and thus are without the remotest possibility of commercial intent,
except to promote THE game, the Game of Diplomacy, which you all 
should purchase from Hasbro or other duly licensed distributors.
 
Stephen Agar has matched the Hasbro rule lists 
and more with some of the even older
rulebooks.
Check these out if you like:

http://www.hasbro.com/default.asp?xcc\_gameandtoyinstructions

http://www.hasbro.com/instruct/Diplomacy.PDF

http://www.hasbro.com/instruct/Diplomacy(OlderVers).PDF

Nice of them to make BOTH of these available.
And all seven different US rulebooks for Diplomacy can now be found here
courtesy of Stephen Agar (relatively new address for this):

http://www.diplomacy-archive.com/diplomacy\_rules.htm

%1: The Reformer, who seeks perfection (crusading type).
%2: The Helper, who seeks love (altruistic type).
%3: The Performer, who seeks status (competitive type).
%4: The Individualist, who seeks an identity (imaginative type).
%5: The Investigator, who seeks knowledge (analytical type).
%6: The Loyalist, who seeks security (ambivalent type).
%7: The Enthusiast, who seeks gratification (thrill-seeking type).
%8: The Leader, who seeks self-sufficiency (dominant type).
%9: The Peacemaker, who seeks contentment (accommodating type).
%See http://graphics.lcs.mit.edu/~becca/enneagram/
%new test at: http://www.9types.com
%My scores in order were: 9, 13, 15, 7, 15, 6, 17, 18, 14.
%Robert Lesco's Marco Poll, the defacto poll of the
%North American hobby since the Runestone is pretty much
%officially dead, continues to collect votes.
%This is an official {\it TAP} approved poll.
%Here are the particulars on the Marco Poll:
%Vote for your top five in each category and you can't vote (or at least
%it won't be tabulated) for Robert Lesco or his szine, which I don't see.
%I will send this to Robert and offer him a trade.
%The deadline is September 1st and either send your ballot to marcopoll of yahoo.com
%or mail to Robert Lesco, 49 Parkside Drive, Brampton, Ontario, CANADA  L6Y 2H1.
%For identification purposes, Robert would like you to identify yourself both
%by name and by affiliation (a game you play in or a szine you sub to).
%Here are the categories this year (I like some of the new ones!!):
%1) Favourite {\it ((Hey, what, he's Canadian, yeah.))} Zine  {\it ((What's a zine,
%shouldn't that be Szine???))}
%Backstabbers United Monthly, Costaguana, The Diplomatic Pouch, Absolute, Vertigo
%2) All Time Favourite Zine {\it ((Szine))}
%Magus, Europa Express, Thirty Miles of Bad Road, Perelandra, North Sealth West George
%3) Best Diplomacy Player
%John Quarto von Tivadar, Edi Birsan, Mark Fassio, Chris Martin, David Partridge
%4) All Time Best Diplomacy Player
%Edi Birsan, Mark Fassio, Kathy Caruso, Rod Walker, Hohn Cho
%5) Best GM
%Malcolm Cornelius, Tim Miller, Tim Richardson, Conrad von Metzke, Doug Massey
%6) Player Who Writes the Best Letters/E-Mails
%Don Williams, Richard Weiss, Mark Fassio, Toby Harris, Cal White 
%7) Folded Zine {\it ((Szine))} You Miss the Most
%Magus, Diplomacy by Moonlight, Le Front du Liberation de Diplomacy, Kathy's Korner
%8) Favourite Variant
%Modern, Nuclear Yuppie Evil Empire Diplomacy, Woolworth, Spy Diplomacy, Colonia

\bigskip

\centerline{Check out current and back issues of {\it Diplomacy World} -- Yahoogroup diplomacyworld}

Also, I need any Hobby Award Nominations NOW (!!!) for:

\noindent The 2005 Don Miller Award for Meritorious Service;

\noindent The 2005 Rod Walker Award for Literature;

\noindent The 2005 John Koning Award for Player Performance;

\noindent The 2005 Fred Hyatt Award for GM Performance;

or

\noindent A 2005 Kathy Byrne Caruso Award for Lifetime Achievement (if warranted).

The Hobby Awards Committee is Jim Burgess (Chair and Treasurer),
Fred Davis, Jr., Melinda Holley, Gary Behnen, Jamie Dreier,
Paul Kenny, Mark Stretch, and Robert Lesco.
I was going to publish the award ballot for this year 
in this issue, but I realized that I really just had to get this out first and
then do that.... and then, well, REAL SOON NOW!!!

\bigskip

{\it Diplomacy World} Issue Deadlines:

\noindent Deadline Spring 2006, Issue \#97: March 1, 2006

\noindent Deadline Summer 2006, Issue \#98: June 1, 2006

Note that Andrew Neumann has taken 
over the lead editorship from Tim Haffey.
Get us articles NOW!!!
The March issue has been delayed but should be out shortly.
That probably means that Issue \#98 will be delayed until Fall.
It is now Fall and not much has happened.
I'm trying to get my general life in order and then I'll deal with these.

%\noindent Final Hobby Awards Ballot for 2004, nominations now to:
%burgess of world.std.com
%
%The qualifications for the Kathy Byrne Caruso Lifetime Achievement 
%Award are that the awardee must have been: (1) Active in the 
%Diplomacy Hobby in at least Three Separate Decades; 
%(2) Multidimensional in their Contributions to the Hobby 
%(e.g. writing, playing, publishing); (3) Taking Retirement or 
%Semi-Retirement from the Diplomacy Hobby; and (4) One of the 
%Hobby�s Unique Personalities Worthy of Being Remembered as Long 
%as THE Game Continues to be Played.  This award need not be awarded
%each year, but only as worthy candidates are identified.  The only
%past honoree is the late Richard Sharp.
%
%The other award categories are self-explanatory:
%
%\noindent Nominees for the 2004 Don Miller Award for Meritorious Service
%
%\noindent Nominees for the 2004 Rod Walker Award for Literary Achievement
%
%\noindent Nominees for the 2004 John Koning Award for Player Performance
%
%\noindent Nominees for the 2004 Fred Hyatt Award for GM Performance
%
%--- --- --- --- --- --- ---

Editorial Board for Diplomacy World:

\noindent Andrew Neumann, andrewneum of gmail.com -- New Lead Executive Editor!

\noindent Tim Haffey, 810 53rd Ave., Oakland, CA 94601 USA; trhaffey of
aol.com -- Ex-Lead Editor and Archives Editor

\noindent Jim Burgess, 664 Smith Street, Providence, RI 02908-4327, USA;
burgess of world.std.com -- Co-Editor and Publisher

\noindent Stephen Agar, 4 Cedars Gardens, Brighton, UNITED KINGDOM  BN1 6YD;
stephen of stephenagar.com -- Webmaster and Non-US Postal

\noindent Rick Desper, 5440 Marinelli Road, \#204, Rockville, MD  20852, USA;
rick\_desper of yahoo.com -- Demo Games

\noindent Dave Partridge, 15 Woodland Drive, Brookline, NH  03033, USA;
rebhuhn of rocketmail.com -- US Postal

\bigskip

ATTENTION: There is a new company that had been doing a new PC Diplomacy game:

http://www.paradoxplaza.com/news.asp?ArticleID=239\&Page$=$News

You all should read the interview in the Spring 2005 movement issue
of the {\it Diplomatic Pouch} (and I mean ALL!!!!!!! of you!!!!!)
that I think you can find at: 

\noindent /Zine/S2005M/

I'm not sure what is going to happen now, but basically everywhere 
except in Russia the game pretty much failed.
The AI was not good (not a huge surprise) and it just required too many
updates to get it to work.
Besides, it really was designed for ``real time'' play and everyone
doesn't want that.
To find out more general information on DIPLOMACY please visit
www.diplomacy-pcgame.com or contact pr of paradoxplaza.com
but I think it is about to wither away and die.
In sum, I think that is too bad.
But the hobby as a whole is going strong with thousands and thousands of
Diplomacy players worldwide.
REALLY!!!

\bigskip

{\bf Bruce Linsey (Mon, 25 Sep 2006 11:22:25 EDT)}

Just happened to notice the remarks about your brother playing in a game that 
you're GMing.  (See? I DO look through the zine now and then!)

There is absolutely nothing wrong with this, since there's no deception 
involved.  I've run games and fantasy sports leagues in which my wife, brother, and 
cousin have played for many years, and I've never even had a complaint of 
favoritism.  (Not even in 1999, when my wife Krissi had a perfect season and was 
the \#1 team in all of Gonzo Football!)  Really, a relative is no different in 
this respect than a close friend, and many of us in the postal gaming 
community have plenty of friends playing in our games.

I wouldn't hesitate to join one of your games which also included your 
relatives.  Just my two cents worth.

\centerline{Bruce, GonzoHQ of aol.com}

{\it ((Thanks, Bruce.
That's the way I see it too.
It's been too long making it difficult for my brother to play here, so we're going for it.
Now, Edi has a comment on the Lepanto issue, surprise, surprise...))}

\bigskip

{\bf Edi Birsan (9/27/06)}
 
Actually there are several reasons that Lepanto is correctly named...

1. The Battle was a Victory of Italy over Turkey

2. The battle featured the triumph of Italian/Western mechanics-skill over
the Turks

3. The central feature of the battle was a rather straightforward
destruction of the Turks nearly head on

4. The battle broke the Turkish control over the eastern seas (though it was
technically not fought in the Eastern Medit as defined on a Diplomacy map it
was considered as being in the Eastern seas at the time.

5. and most important it was fought on same month and day as my birthday
thus being doubly appropriate!
So there take that....

\centerline{Edi Birsan, EdiBirsan of astound.net}

\bigskip

{\bf A NEWS NOTE:} TIM SNYDER, Zine Register editor reports
Michael Edward Snyder was born on Sept 14 and is doing fine!
Sarah is doing well also!

{\it ((I'm hoping that Tim can be convinced to do an update on the Zine Register
sometime soon, but with this, I'm not hopeful....))}
 
\bigskip

{\bf Tim Haffey (Tue, 26 Sep 2006 19:24:52 EDT)}

I noted with interest your notes on the ``Key Opening" which brought back  
memories of a game where that very set of moves was used against me (I was  
Turkey) by Dave Grabar (playing Italy, what else) and Rich Shatto playing Austria. 
It worked very well against me and even Russia (I forget who was playing 
Russia) jumped in on the Turkey dinner and I was out by year 03 or 04,  I 
forget exactly.  This was a ftf game we played well over 30 years ago, in  
Chowchilla, CA.  Never knew that back stabbing move had a name.  
Strange how you can always remember the bad games.

\centerline{``Tim-Bob", TRHAFFEY of aol.com}

{\it ((Indeed, thanks for sharing, Tim.
I agree, I can remember much more clearly my spectacular failures than my (rare)
successes.))}

\bigskip

\centerline{\bf MUSIC AND MOVIES SECTION (WITH COMMENTS ON OTHER ARTS AND SOCIETY)}
     
\smallskip

I want to pick up the music commentary again and run a ``23 tunes'' contest here
(stolen blatantly from Mark Wightman and the late lamented {\bf The Sprouts of Wrath}.
You send me 23 tunes and five or six lines of commentary on each.  I print them
with comments from me 2-3 at a time and then YOU have to guess who submitted each
set of tunes for that issue.  I give ``bonus points" for quality and subtract points
for really lame selections.
Mark, forget how the scoring goes, one point for each person you guess right each issue,
correct?
To enter, you can send them 2-3 at a time, or you can send me the whole list of 23 tunes.
But I think we should be able to get at least ten people playing.
I have an advance commitment from Brad Wilson to be in, so this should be FUN.
When we're done, whoever wins I'll try to figure out how to make a CD of tunes from
the list for them.
If anyone would like to send me a tape or CD of their 23 (which was the original point)
that would be great, but I don't intend to require that.  I will be playing in the
sense that I'll be putting 23 tunes in, and you have to guess me, but I obviously
won't score points.
I'll start things up in a couple of issues once I have enough entrants.
I've got three people in now, I need more of you.
To get started, I need THREE songs, the first turn will be three songs, and then
two for each of the next ten rounds.
Send me three tunes now with a bit of commentary, and we'll see if people can
guess who you are!!!
And already there is a LOT of diversity in defining tunes and styles, so 
BE CREATIVE!!!
I should be able to get a dozen entrants!!!
 
%To encourage voting early, I picked five voters at random to receive five US
%dollars from me (either in cash or paid to someone for Dip stuff or to get
%a free tape).
%Tony Dickinson won the Round 5 prize, so we had three of the six Round 1 voters
%won and one of the three Round 2 postal voters who also got five chances to win.
%That showed the value of getting in early!!
%Voters in Round 1 were Mike Barno(\$5), Rick Desper (\$5), Tony Dickinson (\$5),
%Drew James, Heath Gardner, and John Harrington.
%Round 2 postal voters were John Schultz (\$5) Ian Moore, and Stan Johnson.
%Round 2 E-Mail voters were Scott Morris, 
%Warren Goesle, Peter Sullivan, and Michael Lowrey.
%Round 3 voters were Dick Martin, Richard Weiss, and Rip Gooch.
%Round 4 voters were Roland Sasseville, Jr. (\$5),
%Don Williams, Brent McKee, and Andy York.
%Round 5 voters were Mark Larzelere, Al Tabor, Jody McCullough, and that's it.
%We'll end up with a monster party tape at the end of it that I plan
%to segue and sequence and copy for distribution.
%The result will be a great New Millenium party tape -- we're going for 90 minutes.
%I've also been thinking that I should put this out on CD
%as truly emblematic of the new millenium.  
%I don't have the capability to do that quite yet, but I think
%I might by then.
%Any suggestions (or especially volunteers) on this front will
%be cheerfully accepted and could receive monetary payments!
%The CD way may still occur if someone steps forward and actually does it for me.
%So far, we have ``I Melt With You'' by Modern English; George Gershwin's
%``I got Plenty O' Nuttin' '' from {\it Porgy and Bess} in the 1957 concert
%recording with Ella Fitzgerald finishing off the vocals after Louis
%Armstrong blows and sings through the tune; Duke Ellington performing
%Billy Strayhorn's ``Take the A Train''; Frank Sinatra's ``New York,
%New York''; something from the B-52's;
%the original Van Morrison and Them version of ``Gloria'';
%The (English) Beat's 12 inch version of ``Save It for Later'' 
%ratchets things up to the next level (wherever you put it!);
%Buster Poindexter's ``Hot, Hot, Hot'' keeps you there;
%``Atomic Dog'' by George Clinton blows the doors off,
%and Koko Taylor cleans up singing Willie Dixon's ``Wang Dang Doodle''.
%After I get timings down, I'll choose exactly which songs below make the list.
%
%EXTRA SPECIAL B-52'S BULLPEN: (7) ``Love Shack'';
%``Rock Lobster". {\it ((A tie, a tie!!!!))}
%
%BULLPEN: (9) Nirvana -- ``Smells Like Teen Spirit";
%``Cumberland Blues'' -- the Grateful Dead.
%(8) ``Twistin the Night Away" -- Sam Cooke; ``In Between Days'' -- The Cure;
%``Mannish Boy" -- Muddy Waters; ``Shake, Rattle, \& Roll" -- Big Joe Turner; 
%``Proud Mary'' -- Ike and Tina Turner.
%(7)  ``I Wanna Be Sedated" -- Ramones; ``Twist \& Shout" -- Beatles.
%(6) ``Magic Carpet Ride" -- Steppenwolf;
%``Crossroads'' -- Eric Clapton; ``Play That Funky Music'' -- Wild Cherry;
%``I Fel Good" -- James Brown;
%``Radar Love'' -- Golden Earring; 
%Devo -- ``Whip It"; ``Echo Beach'' -- Martha and The Muffins.
%(5) ``Sing Sing Sing'' -- Benny Goodman; ``Rocking the Casbah'' -- The Clash; 
%Squeeze -- ``If I Didn't Love You"; ``Roadrunner'' -- 
%Jonathan Richman and the Modern Lovers.
%
%SPECIAL ROLLING STONES BULLPEN: (6) ``Paint It Black''. 
%
%SPECIAL TALKING HEADS BULLPEN: (9) ``Take me to the River".
%(4) the entire {\it Speaking in Tongues} record
%(special call for ``Road to Nowhere'').
%
%SPECIAL ALPHABET SONG BULLPEN: (6) ``YMCA" -- Village People. 
%
%SPECIAL DRUG MUSIC BULLPEN: 
%(8) ``Rd, Red Wine'' -- UB40.
%(6) ``Don't Bogart that Joint'' 
%-- Fraternity of Man; ``The Old Dope Peddler'' -- Tom Lehrer.
%(5) ``Cocaine'' -- Eric Clapton's version; ``Casey Jones'' -- the Grateful Dead.
%(4) ``White Rabbit'' -- Jefferson Airplane; ``Love Is the Drug'' -- Roxy Music.

\bigskip
%Brendan Whyte <bwhyte@mscc.huji.ac.il>
%The "Hallelujah" Chorus from Handel's "Messiah"
%The title tells you what the words are, and the middle few bars are 
%so famous, but how many people, apart from those that sing it, know 
%the whole movement? A great little number this, boppy and exciting to 
%sing, to sing along to, or just to tap ones fingers to on the 
%steering wheel while caught in traffic, though with the new bridge to 
%replace the Woodrow Wilson drawbridge in DC should ease the latter at least.
%
%"Libera Me" from Andrew Lloyd Webber's "Requiem"
%The final movement of this impressive modernist-classical piece 
%leaves the boy soprano on his own to hold pitch and rhythm singing 
%about eternal peace, when the organ comes crashing in over him with 
%dissonance and gusto trying to drown him out in chaos, but its waves 
%of noise break and underneath the sweet high "perpertua" carries on. 
%It's a brave young lad who can maintain that, especially in any live 
%performance in front of an audience. But live or recorded, this is 
%great music and a fascinating musical analogy of peace and goodness 
%and light triumphing over noise and chaos.
%
%"Mau Kamu Suka Kamu" by Rano Karno
%An interesting delve into singing by TV presenter Karno comes up 
%trumps with this sugary but fun silly love song in a disco-dangdut 
%style (one could never call Karno himself 'dangdut', he's too 
%respectable!). Mothers and daughters will sing along in the catchy 
%chorus. A difficult track to find outside Java, but worth the effort 
%of hunting down, especially to see the expression on the faces of 
%expat Indonesians when you play it!

{\it ((I'm printing this next article from Mark Lew's blog with his 
permission, and adding some commentary of my own.
I'm going to try to restrain myself and hope that any of you, or Mark
himself, might add further comments.
I'll apologize upfront for using the editor's perogative to scatter my
comments in throughout all of what Mark says.  
If Mark wants to respond and keep his response ``whole", I'll certainly
do that.
And many of you may be interested in providing your own comments on the
discussion.
That would also be fun for next issue, I hope I get at least a FEW comments.
This also was brought to my attention by Bruce Linsey, so thanks to him.
I don't watch Mark's blog as carefully as I might.....
You can find Mark's whole blog at: http://radio.weblogs.com/0134204/index.html ))}

\bigskip

{\bf Mark Lew (Fri, 29 Sep 2006 15:48:57 EDT)}

\smallskip

\centerline{What We Should Learn From the Failure of the Bush Presidency
by Ig Lew}

That's an ambitious title. For almost a year now I've had it in mind to write 
an article that lives up to it. There's a lot that's wrong with the Bush 
presidency, but I've been frustrated by the shallow analysis by many Bush critics 
(mostly Democrats, but not entirely) that boils the whole thing down to, ``Bush 
is bad, so we should replace him with someone good." There's an important 
article to be written rebutting that view.

What follows is not that article. Realistically, I'm never going to get 
around to writing it. I did, however respond to a post in RMO one rare day when I 
had some free time, and some of the related ideas came out. After about five 
paragraphs I realized RMO wasn't the place for it, so instead of posting, I 
saved it for Benzene instead.

The context isn't important. One of RMO's resident liberals mentioned that 
she's looking forward to the end of Bush's presidential term, and along the way 
referred to that as ``a change in regime". 

Regime change doesn't solve the problem. The Bush presidency has been 
damaging to the country not because Bush is incompetent (we've had incompetent 
presidents before), nor because he's evil or malicious (which I don't believe he 
is), nor even because he has political goals that you or I disagree with (we've 
had plenty of that before, too).

We've survived plenty of bad presidencies before. The reason this one 
threatens to do lasting damage is because of structural damage to our political 
system. The United States has a long record of social, economic, and military 
success in large part due to our excellent form of government. Crucial to that 
success is the various correcting mechanisms built into the system. In addition to 
the celebrated checks and balances between our three branches of government, 
there is the balance that comes from reconciling the two sides in our 
two-party system. Most important of all there is the matter of accountability, 
ultimately to the electorate.

{\it ((I would add one comment to that excellence where it has failed, and that
is where one of the parties itself fails or needs to be reborn and I think
that's where the Democratic party is.
I actually think we need a new party that is fiscally conservative and
socially liberal, and this actually would oppose the current Republicans
who are on the opposite side of both of those.
The Republicans have figured out that they can be big spenders if they
can write lots of contracts.
Eventually, we may figure out how to better write government contracts,
and there is lots of innovation going on in this area -- including the
relatively new concept of ``Business Associate" agreements that set up
rules and responsibilities before you even get to bid on a contract -- 
but far too much of the contracting is itself extremely inefficient
and ineffective.
This happens in different ways from the inefficiency/ineffectiveness
in traditional government services, but it really does happen and is
not the panacea Republicans would have us believe.
Another big issue is the failure of trade unionism.
Watch Andy Stern, who is the head of the Service Union and trying to recreate
that institution.
People who I know and really respect that know Stern personally say really
wonderful things about him.
And what is happening to the economy in this country?
It is more and more a service economy.
So the populism we need as the alternative to the government contract
oriented Republicanism is a true respect for diversity, that lets
the best ideas bubble up, not be dictated down.
And that will be socially liberal and fiscally conservative.
I would also note, as a liberal Christian in a fundamentally
religious country (using fundamental in its other sense), that
this is most definitely NOT a give-back to secular humanism.
Our other serious problem in this country is the dramatic
decline in ethics (and if you believe the information on this,
this cuts across liberal religious, conservative religious, and
secular humanists at the individual level) and that also to me is
a problem with top-down ethics and ethics by Directive that this
administration also is trying to foster.
Ethics start with people and communities too.
And we ain't got it right now AND even more worrying, it is getting
MUCH worse in my view.
The best evidence to me is the complete reversal of the ``traffic driving
courtesy" that was so wonderful right after September 11th and is completely
in the toilet now.
He who would be first, shall be last is still a powerful and not fully
embraced dictum.
Watch Andy Stern....))}

All of these are now severely damaged, in many cases deliberately damaged by 
the Bush regime. Power is dangerously concentrated in the executive branch 
now. Current lack of congressional oversight is unprecedented in our history. 
Even when Congress is inclined to participate in government it is crippled by 
lack of good information. The executive branch controls information, and it 
manipulates policymaking by selectively withholding it, distorting it, or in some 
cases even fabricating it. The judicial branch is similarly shut out, with any 
court that disagrees with executive power branded as ``activist". Legislation 
is increasingly ineffective, as the executive department declines to enforce 
laws it doesn't like and fails to implement congressionally mandated government 
initiatives. Recently, the executive repertoire has expanded to redefining 
Congress's laws to mean something other than what Congress explicitly intended.

{\it ((Yes, I see a lot of this first hand. 
One of the reasons this szine is late again this time is that I'm leaving my
current job (where it is just too untenable to me to be told ``not to think
about that, because we don't want anyone to know the answer" from the more usual
``tell me the answer, but I might make decisions politically that ignore that"
which has always been the case in government) to get a bit more freedom of thought
in a related but different job.))}

Partisan power has become dangerously unbalanced as well. Traditions and 
procedures in American government had evolved so that a certain amount of 
bipartisan support was necessary for major decisions. The corrosion of rules in 
Congress began before 2000, but it has grown considerably worse during the Bush 
presidency. Everything we learned in high school civics class about how laws are 
made is obsolete now. Sometimes members of Congress aren't even given an 
opportunity to read the laws they vote on, much less participate in writing them.

Worst of all is the loss of accountability. The greatest strength of our 
political system was the way in which successful policy would drive out 
unsuccessful policy. With the political equivalent of Adam Smith's ``invisible hand", 
those policies and their implementors who succeeded in providing the electorate 
what it wanted would be rewarded while those that failed would not. In a 
healthy administration, the same dynamic occurs within the government as well. When 
a department's program is successful, it is emulated. When a department fails, 
the person responsible is sacked and someone else is given a chance. Even if 
it isn't that person's ``fault" per se, it's still part of the process that 
assures that good ideas rise to the top.

What worries me about Bush's detractors is the naive belief that simply 
removing Bush will solve these problems � that Bush fails simply because he's a bad 
president and if we replace him with a good president, all will be well. It 
won't. Without repair of our political system, the next president � whether 
Democrat or Republican � will be just as ineffective.

{\it ((I think that is sadly, precisely the point.
In my view, it is all about control and power.
In any organization, and this is a very old idea, you don't promote innovation
unless you find good people and free them up to come up with new ideas and 
things.
But if you try to rule through Directive, as this administration does, you 
rely an awful lot on the small number of people writing directives for 
``good ideas and things", plus no one has really figured out how to convince
people to follow most kinds of Directives on issues that really matter.))}

For all our squabbling about hot-button ideological issues, there really 
isn't that much distance between what Democrats and Republicans want. We all want 
economic prosperity, protection from terrorism, disengagement from Iraq 
without the Middle East exploding into a messy war, etc. The disagreement is about 
the best strategy for achieving these goals. Of course these are all difficult 
questions. If they weren't, there wouldn't be the heated debates. We're all 
just taking our best guesses, and no one really knows the answers. If someone 
did know � and again, this is the genius of our (now damaged) political system 
� that answer would be embraced. If someone stumbled on a plan that really 
did, say, bring peace to the Middle East, we wouldn't continue to bicker about 
it. We would say, ``Hey, that really works; by all means let's do it." At worst, 
each party would grab the idea and claim it as its own, but either way it 
would get done.

Bush's detractors seem to think that he fails simply because he's dumb and 
has the wrong ideas and we just need to put in someone else who's smart and has 
the right ideas. Of course, Bush doesn't know all the answers. No one does. 
Our system of government does not require the president to have all the answers, 
and that's exactly why America has proved more successful in almost every way 
than totalitarian states like the Soviet Union.

The reason Bush has failed so spectactularly is that he has destroyed the 
mechanisms whereby good ideas rise to the top. There is no accountability 
anywhere in the administration, all dissenting views are vigorously squelched, and 
all information is kept as confidential as possible. The result is that when 
someone is about to do something stupid � as will inevitably happen � there is 
nothing there to correct the error, either before or after the fact. If some 
piece of our government policy is terribly misguided, there might well be a 
whole lot of people out there who would recognize the problem, but they probably 
don't get a chance to see what's going on, or if they do see it they aren't 
listened to. Once the bad policy is in place, it stays in place � possibly 
because the ill effects are kept hidden, but even after a department head 
accumulates a six-year record of bad results, he's still allowed to continue, because 
this administration values loyalty over competence.

{\it ((Yes and no, actually in my experience the patience actually is much less now
but it is about failure to appear (and appear is the key word) loyal in some
relevant way.... including bad political publicity.
So the key is not whether or not you GET bad results, but whether or not it
becomes generally known that you had bad results.
This was the problem with ``Brownie" for example in Emergency Management of 
Katrina.
I thought it was interesting that the ``big complaint" against him was that quote
about when he knew about people trapped at the Superdome ``today" when he said
that because he hadn't been to sleep for three days and to him three days ago 
WAS today.))}

When thinking about who to elect as president in 2008, voters � both Democrat 
and Republican � who are appalled by the breathtaking failure of the Bush 
presidency should not be thinking about which candidate, or which party, has the 
right policies and the right ideas. What we need to be considering is which 
candidate (or party) is committed to repairing a political system which brings 
good policies to the top regardless of who suggested them. Given that the 
damage to the system was done primarily as a means of pursuing power, both to the 
party and to the executive, undoing the damage is going to require a president 
willing to let go of a lot of power. What we emphatically do not need is an 
anti-Bush whose attitude is, ``The other guys had their chance to do whatever 
they want without restraint; now it's our turn."

{\it ((In academic speak, those of us who talk about this, discuss the need to 
embrace the Third Generation of Knowledge Management (where Bush and his people
are stuck early in the First Generation) where you recognize that some problems
(like Iraq) are inherently complex and need to be approached that way and not
simplified, some problems are complicated but can be simplified, and other problems
really are inherently simple; and context and communities of people are all that
really counts in determining how to solve them.
Thanks, Mark.
I hope I answered your question below.))}

\bigskip

{\bf Mark D. Lew (Sat, 30 Sep 2006 13:49:24 -0700)}

{\it ((Hey Mark, do you mind if I print your review of the political situation
in {\bf TAP}.  I agree with it 100\%, as you know, our political views are
pretty well in synch most of the time.  I agree that the REAL problem
is that bubbling up innovation in information economy issues being
suppressed eventually will kill our economy.  It is, in fact, what
drives our economy.))}

I assume you're talking about the same one Brux just forwarded to his 
mass-email group, right?  The one titled ``what we should learn about 
the Bush presidency", or something like that.

That would be fine. I'm very interested in what you'd have to say about 
the general idea of the free market metaphor as it applies to 
decision-making within a bureaucracy.  There was some of that in John 
Kay's recent book (I forget the title, but you probably know it: 
something very Adam Smith-like), and I found it very inspiring.  But 
rereading my article lately, I didn't really pursue that as much as I 
would have liked.
{\it ((I'm not sure I know precisely that book, but the idea these days
is that people are picking up is Adam Smith's ideas of moral sentiments,
which is partly influencing me in what I said above as well.  A book you
might really like to pick up is Daniel Goleman's book on ``Social Intelligence",
I've not read it yet, but I intend to do so soon.))} 

I think the accountability of bureaucrats within the system is the 
larger problem than general accountability of office-holders to the 
electorate, but it reads like I'm emphasizing the latter more.  Maybe 
I'll follow up on that in the context of the ``Richard Clarke was/wasn't 
fired" debate.

\centerline{mdl, markdlew of earthlink.net}

P.S.  I've moved so much lately that I don't know what address you've 
got, but I'm pretty sure it's wrong.  As of August 2006 I'm at:  1844 N 
190th Street, Shoreline, WA 98133

{\it ((I think that makes sense, but as you saw, I expanded it out into
how we can get accountability of contractors to the government, which
first requires accountability of the bureaucrats who write contracts and
oversee them.))}

\bigskip

{\bf Rick Desper (Wed, 4 Oct 2006 05:25:51 -0700 (PDT))}

Today's game is: given team X, X not the Yankees, how many of the starters for team X 
would be starting for the Yankees?

For the Red Sox, I thought at the beginning of the season the answer 
was three (the obvious two plus Varitek) but after Varitek's weak season 
I think it might be down to two.  
{\it ((Possibly, but I would argue from the ``look what happened when Jason missed
a month" perspective to say that Varitek is still a winner doing what really is
important for a catcher: managing the game on the field and calling the pitches.
He really wasn't all that much weaker at the plate than he's ever been if you
break it down either.))}
This is one of the reasons I got very 
annoyed at people who were filling out the days of September by trashing 
Manny Ramirez.  One guy at a Sox blog said 
``he does this every season", after which I pointed out that 
Manny had played $150+$ games in each of 2003, 2004, and 2005.
So, in case it's not obvious, the Sox starters would be Manny Ramirez and David Ortiz.

For a reference, let's look at the Yankees lineup:
Damon CF,
Jeter SS,
Abreu RF,
Sheffield 1B,
Giambi DH,
ARod 3B,
Matsui LF,
Posada C,
Cano 2B.

There was an amusing bit on ESPN's page 2 yesterday that did this analysis 
for the Tigers, and picked the Tiger at eacch position.  :)  
More seriously, I think from the Tigers I would definitely start Pudge 
over Posada, and I would get Magglio into the lineup somehow.  
{\it ((Pudge is the same way.
Pudge and Manager of the Century Jim Leyland are willing the Tigers to the top.))}

The Twins:
Castillo 2B,
Punto 3B,
Mauer C,
Cuddyer RF,
Morneau 1B,
Hunter CF,
R White LF,
Nevin DH,
Bartlett SS.
Morneau and Mauer are definite starters.  I'd also take Hunter over 
Damon and Cuddyer over Abreu.  Not bad here.

The As -- I started to list the lineup and then it became obvious that Frank Thomas 
would be the only one I'd play.
Ugly.

NL -- 
Cardinals:
Eckstein SS,
Duncan LF,
Taguchi LF,
Pujols 1B,
Edmonds CF,
Rolen 3B,
Encarnacion RF,
Belliard 2B,
Molina C.
It's harder to make this comparison because of the DH discrepancy.  
Pujols would obviously 
start.  Other possibles would be Rolen, Duncan, and Edmonds.  
Edmonds is getting a bit slow with the bat, but he's definitely 
the better fielder than Damon.  Duncan v. Matsui is pretty close.  
Let's go with Duncan because Matsui was hurt most of the season.  
And I'd definitely want to have Rolen at 3B - the question would be 
what to do with ARod?  I would use ARod as the DH in an AL park and 
in the NL park I'd stick him in the outfield or something.   
So I'd take Pujols, Duncan, and Rolen.

Padres:
Piazza's not the hitter he used to be, but Posada never was that hitter.  
I'd take Piazza and that's about it, though I'd want Mike Cameron around 
to sub in for Damon for defense.

Dodgers: 
Russell Martin, Nomar, and Jeff Kent are the candidates.  
Martin might become a better hitter than Posada some day, but not yet.  
Jeff Kent is on the downside of his career and Cano is better now.  
That leaves Nomar.  Nomar doesn't have quite enough hitting to be a top-flight 1B, 
but then Sheffield is not exactly playing in position either.  
I cannot see Nomar dislodging either ARod from 3B or Jeter from SS, 
but I'd want to get his bat in the lineup somehow.  So let's again 
say 1 player and that's it.

Mets:
going with
Lo Duca C,
Delgado 1B,
Valentin 2B,
Reyes SS,
Wright 3B,
Floyd LF,
Beltran CF,
Green LF.
Beltran is in - no brainer.  Lo Duca?  Almost.  Delgado - yeah, I'd start him over 
Sheffield. Floyd? Nope. Green? Nope.
Wright or Valentin?  No way.  That leaves Reyes - an interesting  problem.  
Reyes vs. Jeter would make for an interesting debate.  
I suspect Reyes is the better fielder and, although he doesn't quite 
hit as well as Jeter, he is a great base stealer.  
I think I'd find a way to bump Cano and play Reyes.  
Or DH Jeter, play Giambi at 1B and bump Sheffield.

So that makes 3 Mets, 1 Padre, 1 Dodger, 3 Cardinals, 3 Twins and 1 Athletic.  
I should get back to work...

\centerline{Rick, rick\_desper of yahoo.com}

{\it ((Now for the predictions before the playoffs started.... note how in the NL,
things went to ``Yankee comparison" winners, but NOT in the AL.))}

\bigskip

{\bf From: Warren Goesle (wgoesle of comcast.net)}:
Uh-oh.  I'm going to mostly agree with Rick.  That probably means we're both 
wrong, doesn't it?

{\bf From: Rick Desper (rick\_desper of yahoo.com)}: Let's see what Boob says.  
If he agrees with both of us, that means Dave needs to find an online 
sports gambling site and bet the exact opposite.
p.s. Senator Lieberman still has Sox over Cubs in the World Series!

\bigskip

{\bf Rick Desper (Mon, 2 Oct 2006 10:03:19 -0400 (EDT))}

And now the Red Sox can hibernate until next Spring.
{\it ((But this will be an EXCITING off-season.  I think things will happen now
and we will really see a new team next year.  Sometimes you have to lose
to make the big changes.))}
 
I'm going to jump on the Twins bandwagon.  I remember about the middle 
of the season saying that the Twins would be a nasty opponent in the 
playoffs if they somehow made it in, and since I'm basically stuck 
going with the Yankees, Mets, or Twins, I'll go with the Twins.
 
Looking at the various series:

Yankees v. Tigers:
Well, it was a nice season for the Tigers, but it ends now.  
If you want a justification, compare the relative second-half records.  
The Yankees are still a bit short on pitching, but their hitting is monstrous.  
The Tigers have good pitching, but they've been a beat-up-on-weaklings 
team all season.  Yankees in four. 
{\it ((Yeah, the Yankees will overpower them, Yankees in four sounds about right.))}

A's v. Twins:
Twins have Santana.  A's counter with Zito, who's not been nearly as 
good for quite some time.  Twins also somehow have hitting with Mauer, 
Cuddeyer, and Morneau.  A's are hot but Twins are hotter.  Twins in four. 
{\it ((In my opinion, this is the series for the Champion.  I think the A's will 
win this, really!  Yes, J-B being contrary again, but I feel it in my 
bones and have been thinking the A's would win it all for some weeks now.
This series will go to the limit, and Frank Thomas is hugely hot, and he
will win the series in the last game for the A's.))}

Cardinals v. Padres: Well, the Wells trade worked out for the Padres.  
The Cardinals have been disappointing for the past month, and their 
pitching has never really come together.  I tend to ignore the NL, 
but I've had Pujols and Duncan on my fantasy team this season.  
If those two and Rolen are hitting, the Cardinals definitely have 
the better lineup.  But they still don't have enough pitching.  
I'll go with San Diego in five.
{\it ((I like Duncan too, and he was on MY fantasy team (we think alike!).  But
the Padres will win this.))}

Mets v. Dodgers: Wow, there are a lot of ex-Red Sox in the playoffs!  
Too bad for Pedro that he's going to miss the postseason, and too 
bad for the Mets, too.  If Pedro were healthy, this would be an easy pick.  
I'm still going to go with the Mets, but it could take four or five.  
Let's say five.  
{\it ((And the Dodgers will win this.  The Mets are going to have a lot of 
trouble.  Steve Trachsel has to have a great start for them to win,
and he won't.))}
 
Well, it's time for coffee.  Final pick is Twins over Mets in World Series.
 
\centerline{Rick, rick\_desper of yahoo.com}

{\it ((Dodgers and Padres will be a VERY tough fought series to see who gets to
lose to the A's or Twins.  I don't think the Yankees have a chance with
their pitching to get to the Series.  I think you flip a coin here too,
but I'd go with the Dodgers, I think.  He who laughs last, laughs best.
And no, I didn't agree with you both....))}

\bigskip

{\bf Warren Goesle (Mon, 2 Oct 2006 05:34:58 -0500)}

Uh-oh.  I'm going to mostly agree with Rick.  That probably means we're both 
wrong, doesn't it?

The Tigers are stumbling.  The Yankees are pretty smooth.  Yankees in 3.
The A's have been doing it with mirrors.  The Twins have been doing it with 
mirrors and pitching.  Twins in 4.

Wells still looks like he should have a keg behind him on the mound.  Cards 
are too inconsistent, and not just their pitching.  Padres in 5.
Look it up:  the Mets are 11-12 in games that Martinez started this year. 
They're actually BETTER with him out.  Mets in 4.

Subway Series part II.  Same result as part I.

And on a local note, the Cubs have been hibernating since mid-April.  The 
White Sox showed just how hard it is to win when you're actually supposed to 
win.  They will be angry in 2007.

\centerline{Cheers!  Goz, wgoesle of comcast.net}

\bigskip

{\bf Rick Desper (Tuesday, October 10, 2006 6:35 AM)}
 
Well, my picks were real good weren't they?  My only correct pick was the 
Mets, and I had had the least confidence in them.  I was really hoping they 
would fall, too.  0/4 is more interesting than 1/4.

Anyway, since I already picked the Mets for the WS, I'll stick with that. 
The Cardinals compensated for their thin staff by using Carpenter twice.  If 
they can somehow get Carpenter two starts against the Mets, they'll have a 
chance.  They also have a chance if somebody other than Pujols gets hot at 
the plate.  Neither pitching staff looks all that impressive right now.

The AL is quite the opposite.  Both teams are pitching well, and the Tigers 
looked especially strong in games 2-4 against the Yankees.  As for the 
Twins, since we'd all agreed that Frank Thomas was the only A hitting well, 
I was baffled that the Twins kept pitching to him.  (Also, their defense was 
horrible, and that's baffling.)  Since I'd already picked both the As and 
Tigers to lose, I won't bother to pick the ALCS, but I'll stand by the AL 
over NL pick for the WS.  One interesting aspect of the ALCS is that it pits 
Dave Dombrowski against Billy Beane - two of the most widely respected GMs 
in the business.  While it's not {\bf that} stunning that the As finally won one 
of these postseason series, the entire Tigers story is compelling.  When 
Pudge signed there two years ago, I figured he was just in it for the 
paycheck, and that it was foolish for a last-place team to spend money on 
Pudge of all people, who was likely in the last few years of his career. 
Boy was I wrong!

\centerline{Rick, rick\_desper of yahoo.com}

\bigskip

{\bf Warren Goesle (Tue, 10 Oct 2006 07:32:26 -0500)}

I was also 1 for 4 (I had the same picks Rick had).  Apparently you don't 
have to be playing well going into the playoffs to actually win in the 
playoffs.  While I thought the Mets had enough talent to overcome a 
late-season funk, the Tigers and the Cards were more than a surprise.

I have to stick with the Mets over the Cards.  I think that New York just 
has too much talent for them.  Figure 5 games.

The A's \& Tigers is about as wacky as it gets, but let's remember that it 
wasn't all that far back in this season that Detroit was considered easily 
the best team in baseball, so I'll take them in 5 as well.

No bets from here on the WS...yet.

On another note, I see that someone in Rhode Island has grown a pumpkin 
weighing 1502 pounds.  Jim-Bob, did the State have to evacuate to make room 
for it?

\centerline{Cheers!  Goz, wgoesle of comcast.net}

{\it ((Very, very funny.
We actually are very successful at growing large pumpkins, so this sort
of thing happens a lot, every fall in fact, around here.
The 1500 pounder is especially large, of course.))}

\bigskip

{\bf Rick Desper (Tue, 10 Oct 2006 09:10:10 -0700 (PDT))}

Just learned of an interesting tidbit: all four of the remaining teams in the 
playoffs last won a World Series in the 1980s (St. Louis in '82, Detroit in '84, 
the Mets in '86 [ouch] and the A's in 1989).

If I were a sportscaster, I would say that's pretty ironic.

\centerline{Rick, rick\_desper of yahoo.com}

{\it ((And it's looking like the ones who have not won in the longest are the ones going back,
doesn't it.
I wrote a note to Mark Lew: ``I think the A's are sitting quite pretty right now, except
for the disaster in the middle infield, which doesn't seem to have
hurt them yet.  Are there any reinforcements coming in for the next
round?  It might be Detroit which just improves the A's chances, I
think -- though I think they'd also beat the Yankees.''))}

\bigskip

{\bf Mark Lew (Sat, 7 Oct 2006 10:41:19 -0700)}

I think D'Angelo Jimenez will continue to start at 2B. He's better than
anyone else we've got.  I do think they'll bring someone up for the
next series, but only as a backup.  There's no obvious choice.

I don't know if you heard, but besides Crosby and Ellis, Antonio Perez
also got hurt right at the end of the season.  He was our third-string
infielder (picked up from the Dodgers in the Bradley-Ethier trade) who
was on the big-league team for the whole season hardly ever got a
start.  Our fourth-string infielder was Mike Rouse, who came up for a
couple weeks during the middle of the season and played very well.  In
spite of that, the management must have projected him as having a
modest ceiling, because the last time we needed a spot on the 40-man
roster we put him on waivers (essentially choosing Jimenez over him),
and he was picked up by Cleveland.

That leaves two possible courses for bringing someone up.  One is to
dig down to AA, where we've got Mark Kiger or Kevin Melillo.  Both are
prospects who look mildly promising but nothing spectacular. Neither
would be deemed ready for the majors now were it not for the emergency.

The other idea is Keith Ginter, who was a second-string major-leaguer
for many years before we got a hold of him and sent him down. We got
him at the start of 2005. At that time Ellis was recovering from the
major injury that had him out for all of 2004, so we needed a backup
plan in case he wasn't any good coming back. (Beane was inquiring after
several good 2Bs that off-season, but Ginter was the only deal he
made.)  Ellis turned out OK, so Ginter got demoted.  In 2005, he got
the same treatment Antonio Perez did this year: ie, a guy with major
league experience and a decent record, but we almost never started him;
therefore, he never found his swing, so when he did get to play he was
terrible, so then he went back to the bench and ended the season with
horrible numbers on a small sample size.  For 2006, Ginter spent the
whole year in Sacramento where he did well, and became a free agent at
the end of the season.

That raises an interesting rules situation.  The rules on adding to the
playoff roster are convoluted already, but basically (as I understand
it) if a player is injured, a team is allowed to bring up pretty much
anyone in the system, subject to the League approving it as legitimate
and not abusing the rule with a dubious ``injury" (the ``K-Rod rule").
Ginter raises a question which hasn't come up before so far as I know.
His contract was due to end at the end of the 2006 season.  Since he's
in Sacramento, the season is over for him and he's a free agent.  But
if he's called up, the season isn't over for Oakland and he's not a
free agent yet.  So is he in the system or not?

The A's have already inquired with the League on the question, and the
ruling was that if they sign Ginter to a 2007 contract, they can put
him on the playoff roster, but if they don't they can't.  This creates
an interesting negotiating situation.  Ginter obviously wants to get
back in the big leagues, and since Oakland was keeping him in AAA, he
fully intended to explore the free agent market.  The A's presumably
want to re-sign him for AAA, which according to the League rule would
be sufficient to put him on the playoff roster now.  Even as a backup,
Ginter is a better choice than the AA guys, so the A's need him and he
knows it, so he could try to use the situation to wangle a major league
contract for 2007 that he might otherwise not have gotten.  On the
other hand, signing any sort of contract means he gets to be in the
playoffs now, on a team that has a reasonable chance of going to the
World Series, so he doesn't really want to miss out on that, so the A's
might use that to wangle a minor-league contract that they wouldn't
otherwise have gotten.

My prediction is that Ginter will indeed end up on the roster.  My best
guess is that they sign him to a modest one-year major-league contract
and then probably trade it during the off-season.  Beane only cares
about the price tag, in terms of what he might have to eat and/or
tradeability of the contract.  Making it major-league rather than
minor-league is more a matter of pride, and it looks better for both
sides if he's technically a major-leaguer.  I think Ginter is still the
backup behind Jimenez, though it's likely they give him a chance to
prove himself, and then keep him in if he does well.

\centerline{mdl, markdlew of earthlink.net}

P.S.  One more guy who could conceivably end up at 2B for the A's is
Hiram Bocachica.  He's a veteran utility guy who has bounced back and
forth between big leagues and AAA.  He's been injured a lot in his
career, and he was out for about a year in 2005-06, but he was tearing
up AAA late in the season.  He's been with several teams, most recently
Seattle, from whom we picked him up at the beginning of 2005.  I think
he was considered a possibility for the big leagues then, but since
then he's fallen into the ``AAAA" category.  He's the one who was added
when Perez got hurt, so he's already on the playoff roster now.  He's
one of those versatile players who has played every position but
catcher.  He started out as a 2B in the minors, but has been almost
exclusively outfield for the past few years.  Presumably he's not a
great idea at 2B, but he's at least a fall-back plan.

{\it ((I believe they ended up bringing up Mark Kiger, and nothing
seemed to matter against the Tigers, especially when Frank Thomas
stopped hitting.
I think I have a note from Mark Lew on that to stick on at the bottom here
I gotta just get this szine out, there are scattered other things I 
could still stick in here, but I'd better not try..... this usually
is the place where I write the last notes of the szine, and is so 
again this time.
Apologies to those of you that I'm pushing back to next time.
This last note also fills out the page.))}

\bigskip

{\bf Mark Lew (Wed, 11 Oct 2006 07:15:04 -0700)}

I'm behind on the story, so maybe you already heard.

A's decided to bring up AA guy Mark Kiger, for the reasons I mentioned
in the last email: not likely to play anyway, and he has the best
defense.

Ginter has complained that he wasn't treated well in the process, which
is probably true.  Team gave him a look, decided against, and left him
with nothing.  Ginter can't be entirely happy about spending all of
2006 and half of 2005 in AAA either, so I don't suppose he likes the
A's organization very much.

\centerline{mdl, markdlew of earthlink.net}

{\it ((No, I don't imagine he does.
I still don't think it would have mattered in the actual games, but it was
interesting to ponder.))}

%\bigskip
%
%\centerline{\bf SPY DIPLOMACY (Variant, revised Dec. 1986)}
%
%by Jim Burgess
%
%Rules:
%
%1) Unless otherwise specified, the standard rules of Diplomacy apply.
%
%2) Players receive adjudications for their own armies and fleets, but not
%the armies and fleets of other powers.  Similarly, each player sees only
%his or her own section of the supply center chart.
%
%3) In addition to the standard units, an unit called a "spy" is added.
%
%     A) Each of the seven powers begins with a spy in each home supply center.
%
%     B) Spies are supplied only by home supply centers (i.e. Russia may have
%as many as four spies at a time while other powers are limited to a maximum
%of three).
%
%     C) Spies are "built" and "disbanded" in Winter seasons at the same time
%as other units.
%
%     D) There is no limit to the number of spies that may be present in any
%province and spies may be ``stacked" with armies or fleets (exceptions are
%listed in Rule 6).
%
%4) Spy Reports
%
%     A) Spies report the location and nationality of armies and fleets in
%the province they occupy and all adjacent provinces immediately following
%the standard Winter, Spring, Summer, and Fall seasons (i.e. just prior to
%the spy movement seasons).  Spies can differentiate between units in retreat
%and successful attackers when two units "occupy" a province.
%
%     B) Spies also report the nationality of spies in the province they occupy
%and the presence (but not nationality) of spies in adjacent provinces at the
%same time.
%
%5) Movement of Spies
%
%     A) Spy movement follows the collection of spy reports, so orders to spies
%may be made conditional on any information gathered by spies on opposing troop
%movements or the adjudication of one's own units.
%
%     B) Spies move after each of the following standard seasons: Winter,
%Spring, Summer, and Fall (note that all spy movements PRECEDE Autumn
%retreats), beginning with Winter 1900.
%
%     C) Spies move as armies do with the following exceptions:
%
%          a) Spy movements cannot be stood off by movements of any other unit.
%
%          b) Spies may be convoyed by fleets of any power during any of the
%four movement seasons for spies.  The convoy orders for fleets in these
%seasons must be ordered by their owner in the usual fashion or else the spy
%convoy movement fails of execution.  Fleet convoy orders for spies may be made
%in addition to regular Spring and Fall moves (note, however, that the fleets
%themselves never move during spy movement seasons).
%
%          c) Spies may also be "picked up" and carried by fleets of any
%power. Embarkation and debarkation are moves that must be made during a
%spy movement season and "permission to board" is not required.  Embarkation
%consists of a movement from a land province to an adjacent sea province
%containing a fleet.  Debarkation consists of a movement from a sea province
%containing a fleet to an adjacent land province.  A spy may not "switch
%fleets in midstream".  Once boarded, a spy may only get off by debarking.
%After boarding a fleet, the spy moves with the fleet during regular movement.
%A spy may not debark from a fleet awaiting retreat (e.g. if a fleet carrying
%a spy is dislodged in Spring, the spy may not debark until the spy movement
%season following the standard Summer retreat season).
%
%6) Spy Mortality
%
%     A) This rule will outline the circumstances under which spies may be
%captured or killed.  Spies may be disbanded in Winter seasons at the same
%time as other units to meet the requirement that a power may not have more
%spies than the number of currently held home supply centers and spies are
%annihilated when they are on board a fleet that retreats off the board.
%Otherwise, spies may only be removed from the board (there is no difference
%in the game between capturing and killing a spy, but players should feel
%free to delineate between the two states in writing press) immediately
%following the four spy movement seasons.
%
%     B) Spies killed or captured in a given season do not submit written
%spy reports for that season, though their final movement may have been
%made conditional on that spy report.  Players may never know for sure how
%a spy was lost, just as in real life.
%
%     C) Spies are "killed" by being assassinated by other spies.  When a
%spy moves, he or she may attempt an assassination, specifying the nationality
%of spies to be executed if discovered.  Each spy may hunt for only one
%nationality in each of the four seasons.  The following rules and examples
%will govern the adjudication of assassinations:
%
%          a) A spy may attempt an assassination without moving (for instance
%if the target is in the same province).  In this case, any spies of the target
%nationality that end the spy movement phase in that province are assassinated
%with one exception.  If the spy attempting the stationary assassination is
%himself or herself assassinated by a moving assassin then that spy
%assassinates no one in that season (not even a potential target other than the
%one that assassinates that spy).
%
%          b) A spy attempting a moving assassination (may include a move of
%embarkation or debarkation from a fleet) is always successful in assassinating
%spies of the target nationality that finish the spy movement phase in the
%province that the spy moves to.  This makes "mutual assassinations" possible.
%
%          c) Example: French spy is on board a French fleet in the Channel.
%An English spy is in London.  Both powers specify assassination attempts
%against the spies of the other.  If the English spy embarks on the French
%fleet in the channel (remember that embarkation does not require the
%permission of the fleet being boarded) and the French spy doesn't move, the
%French spy is killed, but the English spy isn't (he may be immediateIy
%captured, however, see part D).  If a German spy is also on board and the
%French spy tried to assassinate the German spy, that operation would fail
%as well. If the French spy debarks to London at the same time, the two spies
%"pass in the night" and neither is killed.
%
%     D) Spies are "captured" by opposing armies and fleets.  Players may
%attempt to capture spies, specifying the nationality of spies to be captured
%if discovered.  Each army and fleet on the board may search for spies of one
%nationality in each of the four seasons.  The following rules and examples
%will govern the adjudication of attempts to capture spies:
%
%          a) Players may specify a nationality to be hunted separately for
%each unit or they may issue blanket searches to be applied to all units.
%
%          b) At the end of the spy movement phase, any spies of the
%nationality specified in the same province are captured (for example, if
%the French player specifies a hunt for English spies on their Channel fleet
%in the above example, then the English spy is captured).
%
%          c) Players are informed when they successfully capture an opposing
%spy.  Note that the "owner" of the spy, however, is not informed of the
%capture.  Communication just is cut.  This rule is the only exception to the
%generaI rule that armies and fleets are blind.  It is possible to play the
%game without allowing capture of spies by armies and fleets if one wishes
%to allow freer movement of spies (i.e. RuIe D may be ignored).
%
%7) It is suggested that the game be adjudicated in four separate seasons
%in the following order:
%
%     A) Spring: (for Spring 1901 this sequence is preceded by a spy movement
%phase (no spy reports are necessary) followed by assassinations or captures)
%
%          a) Standard movement by fleets and armies, adjudication reported
%only for one's own units
%
%          b) Spy report phase, spies collect information
%
%          c) Spy movement phase, spies move according to Rule 4
%
%          d) Spy assassination and capture phase
%
%     B) Summer:
%
%          a) Standard retreats by fleets and armies, adjudication reported
%only for one's own units
%
%          b) Spy report phase, spies collect information
%
%          c) Spy movement phase, spies move according to Rule 4
%
%          d) Spy assassination and capture phase
%
%     C) Fall:
%
%          a) Standard movement by fleets and armies, adjudication reported
%only for one's own units
%
%          b) Spy report phase, spies collect information
%
%          c) Spy movement phase, spies move according to Rule 4
%
%          d) Spy assassination and capture phase
%
%     D) Winter:
%
%          a) Standard Autumn retreats by fleets and armies, adjudication
%reported only for one's own units
%
%          b) Standard Winter builds and removals, also applies to spies
%
%          c) Spy report phase, spies collect information
%
%          d) Spy movement phase, spies move according to Rule 4
%
%          e) Spy assassination and capture phase
%
%\medskip
%
%These rules have been produced by Jim Burgess (with helpful comments from
%Bernie Oaklyn and Dick Martin).  Permission is granted to reproduce and use
%these rules without restrictions.  The author would be interested to hear
%any comments and criticisms regarding any aspect of the play of this variant.
%The rules have been deposited with the North American Variant Bank.  They are
%also available from the author for the cost of postage or free by E-Mail.
%
%\bigskip
%
%\centerline{Wing Rules} 
%
%1.A wing unit can move over both land and water spaces. 
%  
%2.A wing unit can support actions in any space adjacent to the one it occupies. 
%   
%3.A wing can give, receive and cut support in the same way as armies and fleets. 
%   
%4.A wing unit cannot convoy or be convoyed. 
%
%5.A wing unit cannot capture an SC, but instead blockades it: 
%        
%\indent\indent 5.1.A blockade occurs when a wing unit occupies the SC of 
%another player in a fall season. 
%        
%\indent\indent 5.2.The player who owns a blockaded SC does not count 
%it when counting his total number of SCs. 
%        
%\indent\indent 5.3.A blockade ends as soon as the wing unit no longer occupies the SC. 
%        
%\indent\indent 5.4.Builds only take place in the winter phase, just as in games without wings. 

\vfill\eject

\centerline{\bf {\it THE ABYSSINIAN PRINCE} GAMES SECTION}

\smallskip

``So I called up George and he called up Jim, I said let's make a deal.

He said he'd talk to him.  Gonna start a church where you can save yourself,

You can make some noise, When you've got no choice...

You told me useful things, what people think of me, I guess I should thank you.

It's true, then I agree... I'm all alone, I've got no choice,

I'm all alone, I've got no choice."

From ``Got No Choice" by the incomparable Mark Cutler, from the CD
{\it Mark Cutler and Useful Things}.

\smallskip

\indent If you want to submit orders, press, or letters by E-Mail,
you can find me through the Internet system at
``burgess of world.std.com''.
If anyone has an interest in having an E-Mail address listed so people
can negotiate with you by computer, just let me know.
FAX orders to (401) 277-9904 if you let me know in advance to be sure
the fax machine is set up. 

I am continuing to note cut or failed support orders with a 
small ``s'' instead of a capital ``S''.
This will make it easier on the E-Mailed version of the szine to see
what happened, since the italics don't show there.
The italics DO show on the web page just fine.

\indent Standby lists:

Mike Barno, Dick Martin, Brad Wilson, Jack McHugh, Glenn
Petroski, Steve Emmert, Mark Kinney, Vince Lutterbie, Eric Brosius,
Paul Rauterberg, Bob Osuch, Doug Kent, Sean O'Donnell, Vern Parker,
Heath Gardner, Paul Kenny, and Jeff O'Donnell
stand by for regular Diplomacy.
%I am willing to take over for NMR or start new-Diplomacy.
%Vern Parker
%337 Winter Hill Pl.
%Powell, OH 43065
%VSParker of aol.com
%614-402-5139

Let me know if you want on or off these lists, especially OFF.
Standbies get the szine for free and receive my personal thanks.

\bigskip

\centerline{GAME OPENING INFORMATION} 

\smallskip

We've got lots of openings in the subszines, check them out!!!
Contact Rip Gooch directly at xyropedes of canada.com and try to entice him to return.
Rip has been a bit missing in action lately, but I am assured that he
SHALL return.

I'm ready to start a new Breaking Away game, who's interested???
Challenge David Partridge again as he is in, so are 
Brendan Whyte and Alexander Woo.
I also am giving a free spot to Eric Martin.
Anyone else?
Rick Desper is in too.
That leaves us looking for ONE more spot, can I convince someone to take that now???
Please, please, I'd like to get toward starting that next issue!!!

I am willing to open another new game of REGULAR Diplomacy if there is enough interest!!!
Brad Wilson and Sean O'Donnell start the list, who would like to join them???

Also, is there any interest in another game of Nuclear Yuppie Evil Empire 
7x7 Dip?
I know it may be getting tired, but I really like it.
We have Karl Schmit and Sean O'Donnell on the list, let's get seven!!
It's FREE!!!

I also am starting a game of the variant I designed, Spy Diplomacy.
Signups for that are now open.
Bruce Edwards and Eric Ozog are signed up.
The rules were published recently, ask to see them if you missed them!!!

John Harrington is offering to guest GM a game of Office Politics.
Any interest in that??
Let me or John know!
Jody McCullough and Bruce Edwards are interested, anyone else?

And since Colonia is over, Harold Reynolds is looking to start something else.

Also, I am going to design some postal rules for Devil Take the Hindmost,
and we have an opening here:
Bruce Edwards, Mike Barno, and Eoghan Barry are signed up.
Postal rules from me will be forthcoming shortly, on my never ending to-do list.
I will get them in SOON!
I'm more likely to get these things started if I see some interest.....
I've GOT to do this now,
Eoghan is getting tired of waiting....

Right now, the other thing going is the Modern
Diplomacy game with Wings.
Sean O'Donnell, Jeff O'Donnell, Steve Koehler, Art Schleinkofer,
Bob Holt, Rick Desper, Alexander Woo, and Dave Partridge are signed up for that.
I will start it when I get a full complement of players, we only need TWO more!
%Sean's list: Britain, Spain, Germany, France, Egypt, Turkey, Russia, Ukraine, Italy, and Poland.

\bigskip

\centerline{\bf TAKING OVER ISHKIBIBBLE'S REGULAR DIPLOMACY GAMES ((NOT))}

\smallskip

Last chance for anyone who was a subber to {\it Ishkibibble} to join any new
game for free, but this is a ONE TIME only offer, get your request in now.
It is going to expire VERY shortly.
Marc Ellinger and Fred Wiedemeyer already have taken us up on this request.
Come on, let me call by name who qualifies
for this deal..... Karl Schmit, John Power, Tim Snyder,
Dave Partridge, Graham Wilson, and Kevin Wilson.
I especially would like to fill the Modern Diplomacy game that Dave Partridge
already is in.

\bigskip

\centerline{\bf THE PHIL REYNOLDS MEMORIAL: 2006B, Regular Diplomacy}

\smallskip

\noindent {\bf THE DUE DATE FOR SUMMER 1901 IS OCTOBER 28TH, 2006}

\smallskip

\noindent {\bf THE DUE DATE FOR FALL 1901 IS NOVEMBER 18TH, 2006}

\smallskip

\indent {\it Spring 1901}

\noindent AUSTRIA (Burgess): a bud-SER, a vie-BUD, f tri-ALB, 

\noindent ENGLAND (James): a lvp-WAL, f lon-ENG, f edi-NTH.

\noindent FRANCE (Williams): f bre-MID, a par-BUR, a MAR S a par-bur.

\noindent GERMANY (Ellinger): {\it a MUN-bur}, f kie-HOL, a ber-KIE.

\noindent ITALY (Crow): a ven-TYO, a rom-APU, f nap-ION.
%A Apu-Tun, F Ion C A Apu-Tun, A Tyo-Mun

\noindent RUSSIA (O'Donnell): a war-UKR, f sev-RUM, a mos-LVN, 
f stp(sc)-GOB.

\noindent TURKEY (Wiedemeyer): f ank-BLA, a con-BUL, a smy-ARM.

%\bigskip
%
%\indent {\it Supply Center Chart}
%
%\+ AUSTRIA (Burgess):&TRI,VIE,BUD,ser,bul,rum,gre&(has 6, bld 1)\cr
%\+ ENGLAND (James):&EDI,LVP,LON,nwy&(has 4 or 5, even(r:otb) or rem 1)\cr
%\+ FRANCE (Williams):&PAR,BRE,spa,por&(has 5, rem 1)\cr
%\+ GERMANY (Ellinger):&KIE,BER,MUN,hol,den,bel,swe,&(has 6, bld 2)\cr
%\+ &\enskip mar \cr
%\+ ITALY (Crow):&ROM,NAP,VEN,tun&(has 5, rem 1)\cr
%\+ RUSSIA (O'Donnell):&WAR,STP,SEV,MOS&(has 4, even)\cr
%\+ TURKEY (Wiedemeyer):&ANK,SMY,CON&(has 3, even)\cr
%\+ Neutral:&none&(Total=34)\cr
%
\bigskip

\indent {\it Addresses of the Participants}

\noindent AUSTRIA: David Burgess, 101 Laurel Lane, Queensbury, NY  12804

(518) 761-6687, dburgess of glensfallshosp.org

\noindent ENGLAND: Drew James, 3644 Whispering Woods Terrace,
Baldwinsville, NY  13027

(315) 652-1956, kjames01 of twcny.rr.com 

\noindent FRANCE: Don Williams, 27505 Artine Drive, Saugus, CA 91350, 
(661) 297-3947, 

wllmsfmly of earthlink.net or dwilliams of fontana.org (\$5)

\noindent GERMANY: Marc Ellinger, 751 Turnberry Drive,
Jefferson City, MO 65109

ellingermc of aol.com

\noindent ITALY: John Crow, 824 Kinwest Parkway \#101,
Irving, TX  75063, (214) 532-1418

\noindent RUSSIA: Jeff O'Donnell, 1345 Simpson Drive, Hurst, TX  76053
%402 Middle Ave., Elyria, OH  44035-5728,

(440) 322-2920 or (440) 225-9203 (cell, as late as midnight Eastern)

\noindent TURKEY: Fred Wiedemeyer, Box 92010-Meadowbrook RPO, 
Edmonton, ALBERTA  

CANADA  T6T 1N1, (780) 465-6432, wiedem of planet.eon.net 
%cell 780-497-8283; fax 780-468-6432

\bigskip

{\it Game Notes:}

1) Note that Jeff O'Donnell has changed his mailing address, at least temporarily.
I think that means that you should use the cell phone number above as well.
The other weird thing is look at John Crow's zip code and Jeff's new one.
Look CAREFULLY, no, they're not QUITE the same....

2) Note also that we DO have a Summer deadline, even though of course there
are no summer retreats to be made. 
This is a ``press only" season.  
Here's a start at some press below, more is ALWAYS welcome.

\bigskip
 
{\it Press:}

(ENGLAND to EUROPE):  The English people are disturbed at the rumors of 
war coming from the continent.  It is hard to believe that any leader 
would send his people off to war over something as insignificant as 
control of so called ``black dots".  The peace loving people of Britain 
will stay aloof from any conflict (unless dragged in to protect vital 
national interests of course...)

(BERLIN (DIE ZIETLUNG)):  Kaiser Marcus I announced the commencement 
of defensive maneuvers in the West.  ``After the violent incursions from 
the vintners in Burgundy, we have shifted a few light units of brewmeisters 
into the region to replant vines with hops.  Truly this Oktoberfest 
will be one to remember in the glorious annals of Germany."
 
Ministers were dispatched to all powers fully noting that the moves were defensive 
only and evidence no intent to have any hostile action to any other power in Europe.  
``We have the fondest feelings to all our friends in Europe, as long as Burgundy 
and the Low Countries are pacified under a multilateral Prussian peacekeeping 
force, central Europe shall rest at ease."  
 
In other news, fishermen in the Baltic note that the seas are calm and that 
the fish are teeming.  ``We hope there will be no concussions to spook 
the fish this year," stated Georg Bassmaster a local fisherman.  
German military officials immediately responded promising no action in the north 
to disrupt any fishing fleets from any nation, ``Sweden can rest easy under 
Russian protection."

(ENGLAND to FRANCE):  Spies on the continent picked up many indications 
of a move into the Channel.  This move is defensive -- simply the Royal 
Navy on exercises.

(SISKO'S LOG -- UNKNOWN): Something has happened.
One minute I'm fighting Gul Dukat on the fire cliffs of Bajor, then suddenly
I'm here, in my dad's restaurant in New Orleans with Bashir, O'Brian, Odo,
The Major, my dad, and Jadzia Dax.
These are my people but this is not my universe.
For one thing, there are no Klingons in this universe, and another, Jadzia is 
supposed to be dead.
I am totally unfamiliar with the local alien groups.
On the other hand, I could be in worse places than eating Cajun food in my
dad's restaurant.
I'll figure it out later, Sisko out.

(JAMES FAMILY to JEFF IN OHIO):  The METS! METS! Always the METS! We 
won't even use the ``Y" word in this family.

(SYRACUSE to WORLD):  Is it basketball season yet?  Doesn't it seem like 
the college football season has been going on forever...or is that 
just in Syracuse?
{\it ((Hmmm, there are some shared sports teams here, aren't there.....
next issue seems about right for a full fledged ``Big East Preview" issue.
Get your comments on Big East basketball in now.
Any comments on Big East football will be tossed into the wilds of West Virginia
to fend for themselves.))}

\bigskip

\centerline{\bf SPIRALS OF PARANOIA: 2005A, Regular Diplomacy}

\smallskip

\noindent {\bf THE DUE DATE FOR WINTER 1904 IS OCTOBER 28TH, 2006}

\smallskip

\noindent {\bf THE DUE DATE FOR SPRING 1905 IS NOVEMBER 18TH, 2006}

\smallskip

\indent {\it Fall 1904}

\noindent AUSTRIA (Rauterberg): {\it a BUD-rum}, a VIE S a gal-boh, 
a SER S a bud-rum, a gal-BOH, a gre-NAP, 

{\it a TRI s ITALIAN a ven-tyo} (nso), f ION C a gre-nap.

\noindent ENGLAND (Wiedemeyer): f nat-MID, {\it a WAL-lvp}, 
f ENG S FRENCH f mid-bre, {\it f NWG-edi}.

\noindent FRANCE (Tretick): f mid-BRE, f cly-EDI, a LVP h, f IRI S a lvp.

\noindent GERMANY (Ozog for Tallman): a mun-SIL, {\it a HOL-bel, 
a boh-gal} (d r:mun,otb), 

{\it a tyo-ven} (d r:mun,otb), 

f NWY h, {\it a MAR-pic} (imp), {\it a PIC-bel}, f NTH S FRENCH f cly-edi.

\noindent ITALY (O'Donnell): f tus-PIE, a pie-TYO, a VEN S a pie-tyo, 
f ADR S a ven.

\noindent RUSSIA (Sundstrom): a UKR S a sev-rum, 
{\it a SEV-rum}, a war-GAL, a LVN h.

\noindent TURKEY (Biehl): f AEG S f bul(sc)-gre, f bul(sc)-GRE, f smy-EAS.

\bigskip

\indent {\it Supply Center Chart}

\+ AUSTRIA (Rauterberg):&TRI,VIE,BUD,ser,bul,rum,nap&(has 7, even)\cr
\+ ENGLAND (Wiedemeyer):&LON&(has 4, rem 3)\cr
\+ FRANCE (Tretick):&PAR,BRE,spa,por,lvp,edi&(has 4, bld 1, PLAYS ONE SHORT)\cr
\+ GERMANY (Ozog/Tallman):&KIE,BER,MUN,hol,den,bel,swe,&(has 6 or 7, bld 3(r:otb) or 2)\cr
\+ &\enskip mar,nwy \cr
\+ ITALY (O'Donnell):&ROM,VEN,tun&(has 4, rem 1)\cr
\+ RUSSIA (Sundstrom):&WAR,STP,SEV,MOS&(has 4, even)\cr
\+ TURKEY (Biehl):&ANK,SMY,CON,gre&(has 3, bld 1)\cr
\+ Neutral:&none&(Total=34)\cr

\bigskip

\indent {\it Addresses of the Participants}

\noindent AUSTRIA: Paul Rauterberg, 3116 W.~American Dr., 
Greenfield, WI  53221, 

(414) 281-2339 (E-Mail)  trauterberg of wi.rr.com

\noindent ENGLAND: Fred Wiedemeyer, Box 92010-Meadowbrook RPO, 
Edmonton, ALBERTA  CANADA  T6T 1N1, 

(780) 465-6432, wiedem of planet.eon.net 
%cell 780-497-8283; fax 780-468-6432

\noindent FRANCE: Buddy Tretick, 5023 Sewell's Pointe Way,
Fredericksburg, VA  22407, (540) 898-3386

cell (540) 226-5571 (E-Mail) berniebuddy32 of aol.com

\noindent GERMANY: Terry Tallman, PO Box 782, Clinton, WA  98236,
(360) 331-5698 (\$2) 

terryt of whidbey.net

\noindent GERMANY: Temporary Standby is Eric Ozog, PO Box 1138, Granite Falls, WA  
98252-1138,

(360) 691-4264, ElfEric of Juno.com

\noindent ITALY: Jeff O'Donnell, 1345 Simpson Drive, Hurst, TX  76053
%402 Middle Ave., Elyria, OH  44035-5728,

(440) 322-2920 or (440) 225-9203 (cell, as late as midnight Eastern)

\noindent RUSSIA: Matt Sundstrom, 1760 Robincrest Lane South, Glenview, IL  60025, 
(847) 729-1882, 

Matt.Sundstrom of bbdoch.com or mattandzoe of earthlink.net

\noindent TURKEY: John Biehl, \#8 -- 11530 84th Avenue, Delta, BRITISH COLUMBIA,
V4C 2M1  CANADA, 

(604) 816-0460 (cell) (\$7); jrb of dccnet.com

\bigskip

{\it Game Notes:}

1) Terry is still having some medical problems, 
Eric Ozog is continuing to
negotiate and submit orders for Terry.
Eric originally brought Terry into the Diplomacy hobby all those many
years ago, and so I'm glad he's agreed to do this.

2) Note that Jeff O'Donnell has changed his postal address, at least temporarily.
And you should use his cell phone number above if you want to call him, I think.

3) Obviously, both German armies cannot successfully retreat to Munich, so I've
treated it as one retreat and one annihilation in the SC chart.

\bigskip
 
{\it Press:}


\bigskip

\centerline{\bf FLIP FLOP: 2003G, Regular Diplomacy}

\smallskip

\noindent {\bf THE DUE DATE FOR SUMMER 1908 IS OCTOBER 28TH, 2006}

\smallskip

\noindent {\bf THE DUE DATE FOR FALL 1908 IS NOVEMBER 18TH, 2006}

\smallskip

\indent {\it Spring 1908}

\noindent AUSTRIA (Wiedemeyer): f ADR S a tri-alb, a BUD S a gal-rum, 
{\it a GAL-rum},  

a UKR S a gal-rum, a mos-SEV, {\it a TRI-alb}.

\noindent ENGLAND (Schmit): {\it a bur-mun} (d r:bel,otb), a hol-KIE, f eng-MID, 
a RUH S a hol-kie,

f bel-NTH, f NWG S f bel-nth, f nth-HEL.

\noindent FRANCE (Jeff O'Donnell): a par-BUR, a bre-PIC, {\it f SPA(SC)-mar, 
a GAS-mar}.

\noindent GERMANY (Sundstrom): a BER S a mun, f hel-HOL, 
{\it f SWE-den, a DEN-kie}, 

a stp-NWY, a MUN S FRENCH a par-bur, f SKA S a stp-nwy.

\noindent TURKEY (Levinson): a ank-ARM, f con-AEG, a BUL S a rum, f ion-APU, 
f rom-TYH, {\it a ALB-tri}, 

f tyh-ION, f BLA S a rum, {\it a RUM s a ser}, a SER S a alb-tri.

\bigskip
%
%\indent {\it Supply Center Chart}
%
%\+ AUSTRIA (Wiedemeyer):&TRI,VIE,BUD,ven,nap,mos&(has 6 or 7 or 8, even(2 r:otb) or\cr 
%\+ &&\enskip rem 1(1 r:otb) or 2)\cr
%\+ ENGLAND (Schmit):&EDI,LVP,LON,nwy,bel,hol,por&(has 9, rem 2)\cr
%\+ FRANCE (Jeff O'Donnell):&PAR,MAR,spa,BRE&(has 3, bld 1)\cr
%\+ GERMANY (Sundstrom):&KIE,BER,MUN,den,swe,war,stp&(has 5 or 6, bld 2(r:otb) or 1)\cr
%\+ TURKEY (Levinson):&ANK,SMY,CON,bul,sev,ser,gre,&(has 8, bld 2)\cr
%\+ &\enskip tun,rom,rum \cr
%\+ Neutral:&none&(Total=34)\cr
%
%\bigskip

\indent {\it Addresses of the Participants}

\noindent AUSTRIA: Fred Wiedemeyer, Box 92010-Meadowbrook RPO, 
Edmonton, ALBERTA  CANADA  T6T 1N1, 

(780) 465-6432, wiedem of planet.eon.net 
%cell 780-497-8283; fax 780-468-6432

\noindent ENGLAND: Karl Schmit, 1509 O'Keefe Road, DePere, WI  54115, (920) 338-8402, 

diplomacy of new.rr.com (\$4)

\noindent FRANCE: Jeff O'Donnell, 1345 Simpson Drive, Hurst, TX  76053
%402 Middle Ave., Elyria, OH  44035-5728,

(440) 322-2920 or (440) 225-9203 (cell, as late as midnight Eastern)

\noindent GERMANY: Matt Sundstrom, 1760 Robincrest Lane South, Glenview, IL  60025, 
(847) 729-1882, 

Matt.Sundstrom of bbdoch.com or mattandzoe of earthlink.net

\noindent ITALY: Don Williams, 27505 Artine Drive, Saugus, CA 91350, 
(661) 297-3947, 

wllmsfmly of earthlink.net or dwilliams of fontana.org

\noindent RUSSIA: Sean O'Donnell, 1044 Wellfleet Drive, Grafton, OH  44044,
(440) 926-0230, 

sean\_o\_donnell of hotmail.com

\noindent TURKEY: Alexandre Levinson, Beeklaan 504, 2562BP Den Haag
THE NETHERLANDS, don't need phone, 

levinson7 of hotmail.com (\$5)

\bigskip

{\it Game Notes:}

1) Note Jeff's new temporary postal address above.
You also might want to use the cell phone number, as a result, if you 
want to call him.

\bigskip

{\it Press:}

(SEAN--JIM-BOB): I'm not sure if I follow you, but maybe this will help you 
see what I'm doing. As though I was Turkey:
F Tys-Ion,
F Ion-Apu,
F Rom-Tus,
F Blk S A Bul-Rum,
A Bul-Rum,
A Rum-Bud,
A Alb-Tri,
A Ser S A Alb-Tri,
F Con-Aeg,
A Ank-Con.
I'm figuring that Fred is likely to be ordering:
A Mos-Sev,
A Ukr S A Mos-Sev,
A Gal-Rum,
A Bud S A Tri,
A Tri S F Adr-Alb,
F Adr-Alb.
Now, I look and see how I fared. For the fall, I'll look at what I'd do 
as Turkey with what Alexander did do. In S09, I'll flip it and see how I'd 
fare against Alexander.

(JIM-BOB to SEAN): Yes, well, Sean I do think you're still off, but
it is a good exercise to be undertaking.

(CAPTAIN'S LOG): Wow!  The Romulans have decommissioned a troop transport
on the Ferengi front and a front line Galaxy class Bird of Prey 
defending Deep Space 2, which is now defenseless.
We are signaling for the remaining Romulans there to learn how to
speak English.
If the Children of Tomari stay out of the war, I now believe that victory 
will be ours.

(SEAN--BOARD): I agree with Jeff. I honestly don't see how Kurt and Eric 
Ozog, Jeff and I, or Jim and David Burgess playing together in the same game 
is any different than Eric Ozog and Terry Tallman playing in the same game. 
We're playing to win, so I don't understand how it matters.

(CAPTAIN'S LOG): We are now attempting to break out of this universe using
a Protomatter device.
If successful, we will be duplicated.
One group will stay in this universe, the other into a hopefully more friendly
universe.
A Conversation between Kirk and Spock follows.....

(SPOCK): Captain, our attempt to break out of this universe was not as 
successful as our last duplication.
Something went wrong in the Protomatter stream.
Our counterparts in the new universe were probably not duplicates, although
they had a definite Federation signature (Spock hesitates).

(KIRK): What's wrong?

(SPOCK): After the malfunction, the matter stream had an odor to it.

(KIRK): An odor?

(SPOCK): Yes, Captain.

(KIRK): What kind of odor?  (Spock hesitates again)  Out with it!

(SPOCK): Cajun Food.

(KIRK): Cajun Food?

(SPOCK): Cajun Food.

(KIRK): Any idea why?

(SPOCK): The Federation is composed of multiple humanoid species.
There is only one place where you can get Cajun Food, at least initially.

(KIRK): Earth.  I wonder what it all means?

(SPOCK): We probably will never know.....

\bigskip

\centerline{\bf I CAN'T FIND MY MONEY!: 2001F, Regular Diplomacy}

\smallskip

\noindent {\bf THE DUE DATE FOR SPRING 1914 IS OCTOBER 28TH, 2006}

\smallskip

\indent {\it Winter 1913}

\noindent AUSTRIA (Parker): bld a bud; has a BUD, f ADR, f TUN, f TYH, 
a WAR, a LVN, 

f EAS, a PIE, a CON, a VIE, a VEN, a ARM, a SMY, 
a TRI, a MOS, f AEG.
%A Con Sup A Arm-Ank
%A Arm-Ank
%A Smy Sup Arm-Ank
%F Eas- Syr
%F Tyh-Wes
%F Tun Sup F Tyh-Wes
%F Aeg-Ion
%A Bud-Gal

\noindent FRANCE (Kent): rem a par, f bre; has f MID, f WES.

\noindent GERMANY (Wilson): bld a ber, a kie, PLAYS ONE SHORT; 
has a BER, a KIE, a MUN, f ENG, f BEL, 

a BUR, f LVP, a SPA, f POR, a PIC, a STP, 
a MAR, f NAO, a GAS.

\noindent TURKEY (Miller): rem a syr; has a ANK. 

%\bigskip
%
%\indent {\it Supply Center Chart}
%
%\+ AUSTRIA (Parker):&TRI,BUD,VIE,ser,rum,nap,&(has 15, bld 1)\cr
%\+ &\enskip war,mos,sev,ven,rom,bul,gre,smy,con,tun\cr
%\+ FRANCE (Kent):&PAR,BRE&(has 4, rem 2)\cr
%\+ GERMANY (Wilson):&KIE,BER,hol,den,bel,MUN,swe,&(has 12, bld 2 (PLAYS ONE SHORT))\cr
%\+ &\enskip nwy,mar,edi,lon,spa,por,stp,lvp\cr
%\+ TURKEY (Miller):&ANK&(has 2, rem 1)\cr
%\+ Neutral:&none&(Total=34)\cr
% 
\bigskip

\indent {\it Addresses of the Participants}

\noindent AUSTRIA: Vern Parker, 337 Winter Hill Place, Powell, OH 43065,
(614) 402-5139

VernDip of aol.com is preferred

\noindent ENGLAND: Mark Kinney, 4830 Westport Road, Apt D,
Louisville KY  40222

alberich of iglou.com

\noindent FRANCE: Doug Kent, 11111 Woodmeadow Pkwy \#2327, Dallas, TX  75228

dougray30 of yahoo.com 

\noindent GERMANY: Kevin Wilson, 18623 Santa Maria Drive, Baton Rouge, LA  70809,
225-751-3857,

ckevinw1 of cox.net

\noindent ITALY: Formerly was Heath Gardner, metaphorman of gmail.com

\noindent ITALY: Mike Barno, 634 Dawson Hill Road, 
Spencer, NY 14883 

mpbarno of lightlink.com 

\noindent RUSSIA: Rick Desper, 

rick\_desper of yahoo.com

\noindent TURKEY: Tim Miller, 258 New Mark Esplanade, Rockville, MD 20850,

tim of webjudge.net

\noindent GM: Jim-Bob Burgess, 664 Smith Street, Providence, RI  02908-4327,
$+$1 401-351-0287

burgess of world.std.com

\bigskip

{\it Game Notes:}

1) The GAFT and GA-GA draws are both defeated, the GA-GA sallies forth again.
If you have finally become GA-GA over this game, vote yes to this with your Spring
orders.
If you fail to vote, as usual, you veto the proposal.

2) Vern is back, he had E-Mail problems in trying to get orders to me.
Note at the top of the szine, if E-Mail ever bounces from the World account, 
use the gmail account.

\bigskip

{\it Press:}

(FRANCE -- ALL): Regarding the Austrian failure
to mobilize...I believe the Archduke was busy dealing
with the death of the Crown Prince.  It all has
something to do with a magician, Eisenheim the
Illusionist, but details are sketchy.

\bigskip

\centerline{\bf SECRETS: 1999D, Regular Diplomacy}

\smallskip

\noindent {\bf THE DUE DATE FOR SPRING 1922 IS OCTOBER 28TH, 2006}

\smallskip

\indent {\it Winter 1921}

\noindent ENGLAND (Kent): R a kie-RUH; rem f nao; has f ENG, a GAS, 
a RUH, f MID, f NTH, f HOL, f POR.

\noindent FRANCE (Sasseville): has f MAR, f SPA(SC), a BUR.     
%standing orders, yes to ANY draw
                                             
\noindent GERMANY (Barno): PLAYS ONE SHORT; has a BUL.
%votes yes to FREGT, no to EFRT 
%vote YES to FREGT draw

\noindent RUSSIA (Parker): bld f stp(nc); has f STP(NC), f NWY, a WAR, a DEN, a BER, 
f HEL, 

a KIE, a PRU.

\noindent TURKEY (Linsey): rem f tyh; has a ARM, f AEG, a RUM, 
a GAL, a SIL, f WES, 

a UKR, f GOL, a MUN, a TRI, f NAF, 
a TYO, f PIE, a BOH.
%F Wes S F Lyo
%F NAf S F Wes
%F Lyo S F Wes
%F Pie S F Lyo
%A Mun-Ruh
%A Tyo S A Boh-Mun
%A Boh-Mun
%A Sil S A Boh-Mun
%A Gal-Boh
%A Tri H
%A Rum-Bul
%A Ukr-Gal
%F Aeg H
%A Arm-Smy

%\bigskip
%
%\indent {\it Supply Center Chart}
%
%\+ ENGLAND (Kent):&LON,LVP,EDI,bre,por,bel,hol&(has 7 or 8, even(r:otb) or rem 1)\cr
%\+ FRANCE (Sasseville):&PAR,MAR,spa&(has 3, even)\cr
%\+ GERMANY (Barno):&con,bul&(has 1, PLAYS ONE SHORT)\cr
%\+ RUSSIA (Parker):&MOS,STP,nwy,swe,den,ber,WAR,kie&(has 7, bld 1)\cr
%\+ TURKEY (Linsey):&ANK,SMY,rum,gre,&(has 15, rem 1)\cr
%\+ &\enskip bud,nap,ven,tun,sev,rom,tri,vie,ser,mun \cr
%\+ Neutral:&none&(Total=34)\cr
%
\bigskip

\indent {\it Addresses of the Participants}

\noindent ENGLAND: Doug Kent, 11111 Woodmeadow Pkwy \#2327, Dallas, TX  75228

dougray30 of yahoo.com 

\noindent FRANCE: Roland Sasseville, Jr., 38 Bucklin Street, Pawtucket,
RI 02861, (401) 481-4280 (\$0)

roland6 of cox.net  

\noindent GERMANY: Mike Barno, 634 Dawson Hill Road, 
Spencer, NY 14883, (607) 589-4906 

mpbarno of lightlink.com 

\noindent RUSSIA: Vern Parker, 337 Winter Hill Place, Powell, OH 43065,
(614) 402-5139

VernDip of aol.com is preferred

\noindent TURKEY: Bruce Linsey, PO Box 234, Kinderhook, NY
12106

GonzoHQ of aol.com

\bigskip

{\it Game Notes:}

1) We have some new proposals, please vote with your Spring orders.
Proposed are the TREe and the beGREFT draws.
If you think this game is up the TREe or you are so beGREFT that you
cannot continue, vote for one of these draws.
If you fail to vote, they cannot pass.

\bigskip

{\it Press:}
 

\bigskip

\centerline{\bf FINDING THE COMMUNITY: Breaking Away, Designer's Rules}

\smallskip

\noindent {\bf CONGRATULATIONS TO RICK DESPER WHO HUNG ON FOR THE VICTORY!!!}

%\smallskip
%
%\noindent {\bf GAME OVER -- DUE DATE FOR ENDGAME STATEMENTS IS SEPTEMBER 10TH, 2005}
%
%\smallskip
%
%--F--I--N--A--L-- --F--I--N--I--S--H-- --L--I--N--E--
%--S--P--R--I--N--T-- --F--I--N--I--S--H-- --L--I--N--E--
%
%\centerline{\it Turn 16}
%
%\smallskip
%
%\+ Finished:                &Pebble, Homer, Spades (12), Clay (10), El Vez (8),  \cr
%\+                          &\indent Carl Sagan (6), James (4), Paige (2)    \cr

%--F--I--N--A--L-- --F--I--N--I--S--H-- --L--I--N--E--

%\+ 120 (replenish with a 3) &Spades  \cr
%\+ 119 (no replenishment)   &Empty   \cr
%\+ 118 (replenish with a 3) &Clay, El Vez   \cr
%\+ 117 (replenish with a 5) &Carl Sagan, James   \cr
%\+ 116 (replenish with a 7) &Sand, Paige  \cr
%\+ 115 (replenish with a 9) &Hearts, Geri Lee Lewis, Johnny Peso   \cr
%\+ 114 (replenish with a 12)&Marge   \cr
%\+ 113 (replenish with a 13)&Franklin  \cr
%\+ 112 (replenish with a 14)&Wade, Diamonds   \cr
%\+ 111 (no replenishment)   &Empty     \cr
%\+ 110 (replenish with a 3) &Frederick, Clubs, Edmond   \cr

\bigskip
%Sprint Points: 10,8,6,5,4,3,2,1 (final are just double these)
%Movement Order: A/B/C/D; highest card played that turn; highest card in hand

%\indent {\it Addresses of the Participants -- Their Team and Their Cards}

%\smallskip

%\noindent TEAM 1 (Rick Desper): rick\_desper of yahoo.com (55 points, efficiency
%score 13.7\%)

%Team Name: Team Springfield

%\+ A: Homer Simpson &&Finished \cr
%\+ B: Marge Simpson &&(3) 3 3 12  \cr
%\+ C: Bart Simpson  &&Sulks \cr
%\+ D: Lisa Simpson  &&(4) 3 5 6 \cr

%Total Replenishments: $22+63+46+13+15+29+21+17+25+17+18+33+32+10+13+18 = 402$
                
%\smallskip

%\noindent TEAM 2 (Bruce Edwards): kactusjak of ntlworld.com (8 points, efficiency
%score 2.3\%)

%Team Name: Last Again

%\+ A: Zedd   &&(3) 3 3 3 \cr
%\+ B: Omega  &&(4) 3 3 3 \cr
%\+ C: Paige  &&(13) 5 10 7 \cr
%\+ D: James  &&(16) 3 6 5 \cr

%Total Replenishments: $16+37+29+25+12+13+26+24+21+18+16+15+30+23+32+18 = 355$        
   
%\smallskip

%\noindent TEAM 3 (Tom Howell): off-the-shelf of olympus.net (31 points, efficiency
%score 8.0\%)%360-928-9698

%Team Name:  The Soils; Manager:   Boulder

%\+ A:   Clay    &&(7) 3 5 3 \cr
%\+ B:   Silt    &&(4) 4 4 4  \cr
%\+ C:   Sand    &&(10) 3 3 7 \cr
%\+ D:   Pebble  &&Finished \cr

%Total Replenishments: $17+62+63+25+12+13+12+29+27+21+17+31+15+15+13+14 = 386$   
               
%\smallskip

%\noindent TEAM 4 (Dennis Menear): dmenear of wirefire.com (9 points, efficiency
%score 2.2\%)

%Team Name: The Firm

%\+ A: Edmond    &&(5) 3 3 3 3  \cr
%\+ B: Franklin  &&(10) 4 4 13 \cr
%\+ C: Frederick &&(4) 4 3 3 \cr
%\+ D: Wade      &&(8) 3 3 14 \cr

%Total Replenishments: $12+28+32+35+38+25+24+17+29+21+15+26+21+35+26+33 = 417$

%\smallskip

%\noindent TEAM 5 (David Partridge): rebhuhn of rocketmail.com (36 points, efficiency
%score 8.5\%)

%Team Name: It's In The Cards; Manager: The Joker

%\+ A: Spades   &&(7) 3 3 3 3 \cr
%\+ B: Hearts   &&(6) 2 3 9 \cr
%\+ C: Diamonds &&(8) 5 6 14 \cr  
%\+ D: Clubs    &&(5) 3 3 3 \cr
           
%Total Replenishments: $17+55+51+25+30+24+26+20+34+15+15+25+18+20+22+29 = 426$ 

%\smallskip 

%\noindent TEAM 6 (Karl Schmit): diplomacy of new.rr.com (17 points, efficiency
%score 3.6\%)

%Team Name: 25 Dollar Quartet

%\+ A: Carl Sagan      &&(15) 3 9 13 5 \cr
%\+ B: El Vez          &&(15) 13 10 3 \cr
%\+ C: Geri Lee Lewis  &&(13) 9 9 9 \cr
%\+ D: Johnny Peso     &&(13) 1 8 9 \cr

%Total Replenishments: $12+37+22+44+16+27+28+27+32+39+33+20+24+39+49+26 = 475$

\bigskip

{\it Game Notes:}

1) The rules are on the {\it TAP} website in the {\it Tinamou} section.
Ask if you have any questions.
New game start in this, who wants to play again (or for the first time)???
We have five signed up so far, see the list in the game opening section.
I want to start this up soon, so sign up!!!
Can I twist arms on just ONE more of you to join???

%Up above in parentheses is the card you played to get to where you are
%in the field.  The replenishment card is the last card in your list.
%Be careful to note that the card you played (the one in parentheses) is
%not available for you, for next turn.
%Just for fun, I'm going to keep track of total replenishment, by turn, which is a 
%rough measure of how the teams are doing.
%Of course, it is lining up to get across the sprint and final lines in the
%right places that really counts.
%We can calculate an ``efficiency score" later, which will be the ratio
%of scoring points per replenishment.
%If I'm predicting how the future of this will come out, a 10\% score will
%be really tremendous for this measure.

\bigskip

{\it Press:}


\bigskip

{\it Personal Note to You:}

\vfill\eject
\end
